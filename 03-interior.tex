\chapter{Interior Reconstruction}\label{chap:interior}
In this chapter we will use the observation time functions to reconstruct the topological, differential and conformal data of $V$.

\section{Construction of the topology}
The central idea in this section is to find a subspace of the space of all functions from $\sn$ to $\R$, $\R^\sn$ such that for all $q\in V$, $F_q$ is contained in this subspace and convergence in this subspace is equivalent to convergenge in $V$.
A suitable space for this task turns out to be $\mc{C}^\infty(\sn)$, the space of continuous function $F:\sn\to [-T_\sn,0]$ which are smooth on a dense open set in $\sn$ and bounded derivative, endowed with the metric
\[
    d(F,G)\coloneqq  d_\infty(F,G) + \int_{\sn} \lVert dF_a - dG_a \rVert_{g_\sn}da,
\] where $d_\infty(F,G)\coloneqq \max_{a\in \sn}\lvert F(a)-G(a)\rvert$.
Note that by definition of $\mc{C}^\infty(\sn)$ the subset of $\sn$ where $F$ or $G$ are not smooth is a null set, making the integral well-defined.

We can then define the function 
\begin{align*}
    \mathcal{F}:V &\to (\mc{C}^\infty, d)\\
    q&\mapsto F_q
\end{align*} mapping a $q\in V$ to the function $F_q:\sn\to\R$. 

The following argument establishes that the canonical topological structure on $\mc{F}(V)$, i.e. the topology obtained by taking the subspace topology with respect to the topology induced by $d$ on $\mc{C}^\infty$, is the same as the pushforward under $\mc{F}$ of the topology on $V$, making $\mc{F}$ a homeomorphism.
\begin{lemma}\label{lem:Fprop}
The map $\mathcal{F}:V\to\widehat{V}\coloneqq \mathcal{F}(V)$ is a well-defined continuous and bijective.
\end{lemma}
\begin{proof}
First of all we show that $\F:V\to (\mc{C}^\infty,d)$ is well-defined. Let $q\in V$, then $F_q$ is continuous by proposition \ref{prop:fcont}, smooth on a dense open set (namely $\areg(q)$) by proposition \ref{prop:fsmooth}, with bounded derivative by proposition \ref{prop:dfbounded}.

To prove that $\F$ is continuous we let $q_n\to q_0 \in V$. By proposition \ref{prop:funif} $F_{q_n}\to F_{q_0}$ uniformly and thus $d_\infty(F_{q_n},F_{q_0})\to 0$. Now we need to show that 
\[
    \int_{\sn} \lVert \left.dF_{q_n}\right\rvert_a - \left.dF_{q_0}\right\rvert_a \rVert_{g_\sn}da \to 0.
\]
To that end let $\varepsilon>0$ and $\delta_1$ such that $\cl{B_{\delta_1}(q_0)}\subset V$. Because $q_n\to q_0$, after possibly discarding finitely many $q_n$ we may assume $q_n \in B_{\delta_1}(q_0)$. 
Because $\left.dF_{q}\right\rvert_a$ is bounded on $D_{\delta_1}$ by proposition \ref{prop:dfbounded} there exists a $c>0$ such that $\lVert \left.dF_{q_n}\right\rvert_a - \left.dF_{q_0}\right\rvert_a \rVert_{g_\sn}<c$ for all $n\in \N$ and $a\in \areg(q_n)\cap \areg(q_0)$. 

On the other hand because $\areg(q_0)$ is dense and open in $\sn$ we have $\int_{\areg(q_0)}da=\int_{\sn}da$. Hence we can find an open set $A\in \sn$ such that $\cl{A}\subset \areg(q_0)$ and 
\[
    \int_{\sn \setminus \cl{A}}da< \frac{\varepsilon}{2c}.
\]
Applying proposition \ref{prop:dfunifconv} to $A$ yields a $N\in \N$ such that for all $n\ge N$ we have $F_{q_n}\rvert_{\cl{A}}$ smooth and $\left\lVert dF_{q_n}\rvert_{a} - dF_{q_0}\rvert_{a}\right\rVert_{g_{\sn}}<\frac{\varepsilon}{2}$ for all $a\in \cl{A}$.

We can now write 
\begin{align*}
    \int_\sn \lVert \left.dF_{q_n}\right\rvert_a - \left.dF_{q_0}\right\rvert_a \rVert_{g_\sn}da &= \int_{\sn\setminus\cl{A}}\lVert \left.dF_{q_n}\right\rvert_a - \left.dF_{q_0}\right\rvert_a \rVert_{g_\sn}da\\ &+ \int_{\cl{A}}\lVert \left.dF_{q_n}\right\rvert_a - \left.dF_{q_0}\right\rvert_a \rVert_{g_\sn}da\\
    &<\varepsilon
\end{align*} because 
\begin{align*}
    \int_{\sn\setminus\cl{A}}\lVert \left.dF_{q_n}\right\rvert_a - \left.dF_{q_0}\right\rvert_a \rVert_{g_\sn}da \leq \int_{\sn \setminus \cl{A}}cda &< \frac{\varepsilon}{2} \quad \text{ and}\\
    \int_{\cl{A}}\lVert \left.dF_{q_n}\right\rvert_a - \left.dF_{q_0}\right\rvert_a \rVert_{g_\sn}da < \int_{\cl{A}}\frac{\varepsilon}{2}da &\leq \frac{\varepsilon}{2}.
\end{align*}
Because $\varepsilon>0$ was arbitrary we get $\int_{\sn} \lVert \left.dF_{q_n}\right\rvert_a - \left.dF_{q_0}\right\rvert_a \rVert_{g_\sn}da \to 0$ and thus $d(F_{q_n},F_{q_0})\to 0$ proving that $\F$ is continuous.

Finally, injectivity follows from the fact that for any $q,q'\in V$ we have $\F(q)=\F(q')\implies F_q = F_{q'} \implies \ee(q) = \ee(q')$ which implies $q = q'$ by proposition \ref{prop:einj}.
\end{proof}

However there is still some work required to show that $\F^{-1}$ is continuous on $\widehat{V}$:
\begin{lemma}\label{lem:frevcont} Let $F_n \to F_0$ in $\widehat{V}$ then $q_n\coloneqq \F^{-1}(F_n) \to q_0\coloneqq \F^{-1}(F_0)$.
\end{lemma}
\begin{proof}
    Note that by the previous result $\F:V\to \widehat{V}$ is a bijection and thus $q_n$ and $q_0$ are well defined and we have $F_n = F_{q_n}$ resp. $F_0 = F_{q_0}$. We now aim to find $a_1,a_2\in \areg(q_0)$ such that $dF_{q_n}\rvert_{a_i}\to dF_{q_0}\rvert_{a_i}$, allowing us to apply proposition \ref{prop:dfconvimplqconv}: Let for some set $S\subset \sn$ we let $\mu(S)\coloneqq  \int_{S}da$ be the standard set measure and $S^c = \sn \setminus S$ the complement.

    Let \[
        A\coloneqq \areg(q_0) \cap \bigcap_{n=1}^{\infty}\areg(q_n)
    \] and
    \[
        C = A^c, \quad C_n = \areg(q_n)^c, \quad C_0 = \areg(q_n)^c.
    \]
    Because $\mu(\areg(q_n))=\mu(\areg(q_0))=\mu(\sn)<\infty$, we have $\mu(C_n)=\mu(C_0)=0$. This yields 
    \[
        \mu(C)= \mu\left(C_0 \cup \bigcup_{n=1}^{\infty}C_n\right) \leq \mu(C_0) + \sum_{n=1}^\infty \mu(C_n) = 0
    \] and thus $\mu(A)=\mu(\sn)-\mu(C)=\mu(\sn)>0$.

    We then define the set of \emph{stragglers} as 
    \[
        S(A)\coloneqq \{a\in A \mid \lim_{n\to \infty} dF_{q_n}\rvert_a \neq dF_{q_0}\rvert_a\}.
    \] Because $F_{q_n}\to F_{q_0}$ with respect to $d$ we must have $\int_{\sn} \lVert \left.dF_{q_n}\right\rvert_a - \left.dF_{q_0}\right\rvert_a \rVert_{g_\sn}da \to 0$ which implies $\mu(S(A))=0$. But now we have $\mu(A\setminus S(A)) > 0$ which implies that there exist two $a_1,a_2\in A\setminus S(A)$. By definition $F_{q_n}$ is smooth at $a_i$ for all $n\in \N$ and $dF_{q_n}\rvert_{a_i} \to dF_{q_0}\rvert_{a_i}$. Now we can apply proposition \ref{prop:dfconvimplqconv} and get $q_n\to q_0$ as desired.
\end{proof}
% \begin{proof}
%     Let $(q_n)_{n=1}^\infty, q_0 \in V$ such that $F_{q_n}\to F_{q_0}$ uniformly. Note that because $\F$ is injective the choice of $q_n$ for $F_{q_n}=\F(q_n)$ is unambigous.

%     Let $t_0 = \max_{a\in \sn} F_{q_0}(a)$, because $q_0\in V$, lemma \ref{lem:Kcharact}(3) implies that $t_0<1$. Hence there exists an $\varepsilon>0$ such that $t'\coloneqq t_0 + \varepsilon < 1$. And because $F_{q_n}$ to $F_{q_0}$ uniformly there exists a $N\in \N$ such that $t_n\coloneqq \max_{a\in \sn} F_{q_n}(a)<t'$ for all $n \ge N$. 

%     We now define 
%     \[
%         C_{t'}\coloneqq \{q\in \jp \mid \max_{a\in \sn} F_q(q)\leq t'\}   
%     \] and after removing the first $N$ elements of $(q_n)_{n=1}^\infty$ we may assume $q_n, q_0\in C_{t'}$.

%     We now show that $C_{t'}\subset \jp$ is closed: Let $p_n \to p_0 \in \jp$ with $p_n \in C_{t'}$. We must have $p_0\in \interior{\jp}$ because otherwise $\max_{a\in \sn} F_{p_n}(a)\to 1 > t'$. 
%     We then apply proposition \ref{prop:funif} to $V_0$, to get $F_{p_n} \to F_{p_0}$ uniformly. This in turn implies $\max_{a\in \sn}F_{p_n} \to \max_{a\in \sn}F_{p_0}$ and because $p_n\in C_{t'}$ we have $\max_{a\in \sn}F_{p_n}\leq t'$ for all $n$ and hence also $\max_{a\in \sn}F_{p_0}\leq t'$. This proves $p_0\in C_{t'}$ making $C_{t'}$ closed.

%     Now $C_{t'}$ is a closed subset of a compact space and thus itself compact. Again using $V_0$ and the previous lemma we get that $\F:V_0 \to \F(V_0)$ is well-defined continuous and injective. We can then restrict $\F$ to $C_{t'}$ and all these properties are preserved. But now $\F:C_{t'}\to \F(C_{t'})$ is a continuous, injective map from a compact space to a hausdorff space, making it a homeomorphism. 
    
%     Because $q_n,q_0\in C_{t'}$ for all $n$, we have $F_{q_n}=\F(q_n)\to F_{q_0} = \F(q_0)\in \F(C_{t'})$. Using that $\F^{-1}$ continuous on $\F(C_{t'})$ we get
%     \[
%         q_n = \mc{F}^{-1}(F_{q_n}) \to \mc{F}^{-1}(F_{q_0}) = q_0,
%     \] as desired.
% \end{proof}

% \begin{proof}
%     Let $(q_n)_{n=1}^\infty, q_0 \in V$ such that $F_{q_n}\to F_{q_0}$ uniformly. Note that because $\F$ is injective the choice of $q_n$ for $F_{q_n}=\F(q_n)$ is unambigous.

%     Let $t_0 = \max_{a\in \sn} F_{q_0}(a)$, because $q_0\in V$, lemma \ref{lem:Kcharact}(3) implies that $t_0<1$. Hence there exists an $t'$ such that $t_0 < t' < 1$. Now we endow $M$ with and arbitrary metric compatible with the topology and let $\varepsilon>0$ such that $\overline{B_\varepsilon(q_0)}\subset V$
    
%     % And because $F_{q_n}$ to $F_{q_0}$ uniformly there exists a $N\in \N$ such that $t_n\coloneqq \max_{a\in \sn} F_{q_n}(a)<t'$ for all $n \ge N$. 

%     We now define 
%     \[
%         C_{t'}\coloneqq \{q\in \overline{B_\varepsilon(q_0)} \mid \max_{a\in \sn} F_q(a)\leq t'\}   
%     \] 
    
%     % and after removing the first $N$ elements of $(q_n)_{n=1}^\infty$ we may assume $q_n, q_0\in C_{t'}$.

%     We now show that $C_{t'}\subset V$ is closed in $\jp$ and thus compact: Let $p_n \to p_0 \in \jp$ with $p_n \in C_{t'}$. Because $p_n\in \overline{B_\varepsilon(q_0)}$ is closed, we have $p_0\in \overline{B_\varepsilon(q_0)}\subset V \subset \ijp$.
%     We then apply proposition \ref{prop:funif}, to get $F_{p_n} \to F_{p_0}$ uniformly. This in turn implies $\max_{a\in \sn}F_{p_n} \to \max_{a\in \sn}F_{p_0}$ and because $p_n\in C_{t'}$ we have $\max_{a\in \sn}F_{p_n}\leq t'$ for all $n$ and hence also $\max_{a\in \sn}F_{p_0}\leq t'$. This proves $p_0\in C_{t'}$ making $C_{t'}$ closed.

%     Now $C_{t'}$ is a closed subset of a compact space and thus itself compact. Using the previous lemma we get that $\F:C_{t'} \to \F(C_{t'})$ is well-defined continuous and injective. But now $\F:C_{t'}\to \F(C_{t'})$ is a continuous, injective map from a compact space to a hausdorff space, making it a homeomorphism. 
    
%     Furthermore 
%     Because $q_n,q_0\in C_{t'}$ for all $n$, we have $F_{q_n}=\F(q_n)\to F_{q_0} = \F(q_0)\in \F(C_{t'})$. Using that $\F^{-1}$ continuous on $\F(C_{t'})$ we get
%     \[
%         q_n = \mc{F}^{-1}(F_{q_n}) \to \mc{F}^{-1}(F_{q_0}) = q_0,
%     \] as desired.
% \end{proof}

And we get:
\begin{corollary}\label{cor:Fhomeo}
    $\F:V\to \widehat{V}$ is a homeomorphism.
\end{corollary}

Note that because $\mc{P}_K(V)$ determines all $F_q$ and thus the set $\widehat{V}$, it is uniquely determined by the data \ref{rmk:data}. We can thus reconstruct the topology of $V$ because we can determine the topology of $(\mc{C}^\infty(\sn),d)$ and the subspace topology on $\widehat{V}$ which must be equivalent to the topology on $V$.

\section{Smooth Reconstruction}
Having established the topological structure of $V$ we next aim to establish coordinates on $\F(V)$ near any $\F(q)$ that make $\F(V)$ diffeomorphic to $V$.

\subsection{Construction of smooth coordinates}
In line with the last section we will consider $\widehat{V}\coloneqq \mc{F}(V)$ a topological space. We denote the points of this manifold by $\widehat{q}\coloneqq \mc{F}(q)=F_q$. This means that points in $\widehat{V}$ are functions from $\sn$ to $\R$.
Next we construct a differentiable structure on $\widehat{V}$ that is compatible with that of $V$ and makes $\F$ a diffeomorphism.

This definition is related to the coordinates constructed in proposition \ref{prop:observationtimecoordinates}, i.e. obervation coordinates consist of a tuple of observation times from $n+1$ observers.
\begin{definition}[Observation Coordinates]
Let $\widehat{q}=\F(q)\in \widehat{V}$ and $\aa = (a_j)^n_{j=0}\subset (\sn)^{n+1}$ with $p_j = \mc{E}_{a_j}(q)$ such that $p_j\in \er(q)$ for all $j=0,\dots,n$. Let $s_{a_j} = f_{a_j} \circ \mathcal{F}^{-1}$ and $\mathbf{s}_{\overrightarrow{a}}=\mathbf{f}_{\overrightarrow{a}}\circ\mc{F}^{-1}$. Let $W\subset\widehat{V}$ be an open neighborhood of $\widehat{q}$. We say that $(W,\mathbf{s}_\aa)$ are $C^0$-observation coordinates around $\widehat{q}$ if the map $\mathbf{s}_\aa:W\to \R^n$ is open and injective. Also we say that $(W,\mathbf{s}_\aa)$ are $C^\infty$-observation coordinates around $\widehat{q}$ if $\mathbf{s}_{\overrightarrow{a}}\circ\mc{F}:\mc{F}^{-1}(W)\to \R^n$ are smooth local coordinates on $V\subset M$.
\end{definition}
Note that by the invariance of domain theorem, $\mathbf{s}_{\overrightarrow{a}}:W\to \R^n$ is open if it is injective.
Although for a given $\overrightarrow{a}\in (\sn)^{n+1}$ there might be several sets $W$ for which $(W,\mathbf{s}_{\overrightarrow{a}})$ form $C^0$-observation coordinates to clarify the notation we will often denote the coordinates $(W,\mathbf{s}_{\overrightarrow{a}})$ as $(W_{\overrightarrow{a}},\mathbf{s}_{\overrightarrow{a}})$. 

Crucially for a tuple $\overrightarrow{a}\in (\sn)^{1+n}$ we can determine $\mathbf{s}_{\overrightarrow{a}}:\widehat{V}\to \R^{n+1}$ using only the data from \ref{rmk:data}. This is because for a given $\widehat{q}=F_q\in \widehat{V}$ we have $\mathbf{s}_{\overrightarrow{a}}(\widehat{q}) = (F_q(a_0),\dots,F_q(a_n))$ requiring no external information.

We can now determine the differential structure of $V$:
\begin{proposition}\label{prop:findsmoothcoords}
Let  $\widehat{q}\in \widehat{V}$ then the following holds:
\begin{enumerate}[label={\textnormal{(\arabic*)}}]
    \item Given the data from \ref{rmk:data} we can determine all $C^0$-observation coordinates around $\widehat{q}$,
    \item there exist $C^\infty$-observation coordinates $(W_\aa,\mathbf{s}_\aa)$ around $\widehat{q}$ and
    \item given any $C^0$-observation coordinates $(W_\aa,\mathbf{s}_\aa)$ around $\widehat{q}$, the data \ref{rmk:data}, allows us to determine whether they are $C^\infty$-observation coordinates around $\widehat{q}$.
\end{enumerate}
\end{proposition}

\begin{proof}
    We begin with some setup: Let $q\in V$. We say that $p\in \er(q)$ and $a\in \sn$ are \emph{associated} with respect to $q$ if $p\in \mu_a$, i.e. $p=\mc{E}_a(q)$. 
    
    To prove part (1), we let $\widehat{q}\in \widehat{V}$ with $\widehat{q}=\F(q)$. We want to show that for any choice of observers $\aa=(a_j)^n_{j=0}\in (\sn)^{n+1}$ we can determine if they form $C^0$-observation coordinates. First of all we need to check whether the associated $p_j=\mc{E}_{a_j}(q)$ are regular points, i.e. $p_j\in \er(q)$. But as $\widehat{q}=\F(q) = F_q$ we can recover $\ee{q}=\bigcup_{a\in \sn}\mu_a(F_q(a))$ and also the associated points $p_j = \mu_{a_j}(F_q(a_j))$. By proposition \ref{prop:obsreconstr} this allows us to determine $\er(q)$ and for all $p_j$ we can then simply check whether they lie in $\er(q)$.
    
    We now need to check whether there exists an open neighborhood $W$ of $\widehat{q}$ such that the map $\mathbf{s}_\aa:W\to\R^n$ is injective. 
    By definition we have 
    \[
    \mathbf{s}_\aa(\widehat{q}) = (\widehat{q}(a_1),\dots,\widehat{q}(a_n)) = (F_q(a_0), \dots ,F_q(a_n))
    \] which means that the data allows us to fully determine $\mathbf{s}_\aa$ on $\widehat{V}$. But since by corollary \ref{cor:Fhomeo}, the data allows us do construct the topology on $\widehat{V}$ we can determine whether there exists an open neighborhood $W$ of $\widehat{q}$ such that $\mathbf{s}_\aa:W\to \R^n$ is injective and thus open by the invariance of domain theorem.
    

    To show (2) we let again $\widehat{q}\in \widehat{V}$ with $\widehat{q}=\F(q)$. 
    Let $(a_j)_{j=0}^n \in (\sn)^{n+1}$ such that the associated $p_j \in \er(q)$ and the vectors $\{w_j = w(q,p_j)\mid j=0,\dots n\}$ are linearly independent. We can find such a set of linearly independent vectors because by proposition \ref{prop:submanifolds} $\er(q)$ is an open subset of $\ee(q)$. Now by proposition \ref{prop:observationtimecoordinates} the observation time functions $\mathbf{f}_\aa$ define smooth coordinates on a neighborhood $V_1$ of $q$. Thus $\mathbf{s}_\aa\circ\F$ are smooth local coordinates as well making $(\mathbf{s}_\aa,\F(V_1))$ $C^\infty$-observation coordinates.

    Moving on to part (3): We begin by proving that the set of points in $(\er(q))^{n+1}$ which yield $C^\infty$-observation coordinates is open and dense in $(\er(q))^{n+1}$.
    We consider $p\in \er(q)$ and $a\in \sn$ which are associated. 
    % Recall that equation \eqref{eq:fgradient} implies $f_a(q)$ satisfies
    % \[
    % \nabla_q f_a(q) = \frac{1}{c(q,a)}w(q,p), \quad c(q,a)\neq 0
    % \]
    Let 
    \[
        K(q) = \{(w_j)^n_{j=0} \mid w_j\in L^K_qM, \rho(q,w_j) > 1, \gamma_{q,w_j}(1)\in K\}
    \]
    and define on $K(q)$ the map
    \begin{align*}
        H:K(q)&\to K^{n+1}\\
        (w_j)^n_{j=0}&\mapsto (\gamma_{q,w_j}(1))^n_{j=0}.
    \end{align*}
    We will denote $p_j = \gamma_{q,w_j}(1) = \exp_q(w_j)$. Then by definition $p_j\in \er(q)$ and $w_j = \Omega(q,p_j)$. As $\rho$ is lower semi-continuous, we see that $K(q)\subset (L_q^KM)^n$ is open by an analogous argument to the one in the proof of \ref{prop:submanifolds}. As the exponential map is continuous, $H$ is also continuous. Furthermore as $\Omega:\mc{Z}\to L^+V$ is continuous and injective, we can construct a continuous inverse to $H$, making $H:K(q) \to H(K(q)) = (\er(q))^{n+1}$ a homeomorphism.
    We will denote $Y(q)\coloneqq(\varepsilon_U^{reg}(q))^{n+1}$.
    Note that for all $\widehat{q}\in \widehat{V}$, the data \ref{rmk:data} determine $\er(q)$ and thus also the set $Y(q)\subset K^n$, where $q=\F^{-1}(\widehat{q})$.
    
    Let us now consider the set 
    \[
        K_0(q) = \{(w_j)_{j=1}^n\in K(q) \mid w_1,\dots,w_n \text{ are linearly independent}\}.
    \]
    As linear independence is an open and non-degenerate property $K_0(q)$ is open and dense in $K(q)$. Since $H$ is a homeomorphism, $Y_0(q) = H(K_0(q))$ is open and dense in $Y(q)$ as well.
    
    We can now prove the final part of the proposition: Recall that given $C^0$-observation coordinates around $\widehat{q}$, we want to determine if they are also $C^\infty$-observation coordinates $\widehat{q}$.
    To that end, let $(W_\aa,\mathbf{s}_\aa)$ be $C^0$-observation coordinates around $\widehat{q}\in W_\aa$ with $q=\F^{-1}(\widehat{q})$. By definition we have $p_j\in \er(q)$ where $p_j=\mc{E}_{a_j}(q)$ are associated with $a_j$ and hence $(p_j)_{j=0}^n\subset Y(q)$. In the case where $(p_j)^n_{j=0}\in Y_0(q)$, by proposition \ref{prop:observationtimecoordinates}, $q$ has a neighborhood $V_1\subset M$ on which the function $\mathbf{f}_{\overrightarrow{a}}:V_1\to\R^n$ gives smooth local coordinates. Thus, after possibly restricting $W_{\overrightarrow{a}}$, $(W_{\overrightarrow{a}},\mathbf{s}_{\overrightarrow{a}})$ are $C^\infty$-observation coordinates around $\widehat{q}$. 
    We then let $(W_{\overrightarrow{b}},\mathbf{s}_{\overrightarrow{b}}), \overrightarrow{b}\in (\sn)^{n+1}$ be different $C^0$-observation coordinates around $\widehat{q}$ and let $(\widetilde{p}_j)^n_{j=0}\in Y(q)$ be such that $\widetilde{p}_j$ is associated to $b_j$.
    Since all smooth coordinates must be compatible, then $(\widetilde{p_j})^n_{j=0}\in Y_0(q)$ if and only if 
    \begin{multline}\label{eq:compatibility}
        \text{The function $\mathbf{s}_{\overrightarrow{b}}\circ \mathbf{s}_{\aa}^{-1}$ is smooth at $\mathbf{s}_{\overrightarrow{a}}(\widehat{q})$ and the Jacobian determinant}\\
        \det(D(\mathbf{s}_{\overrightarrow{b}}\circ \mathbf{s}_{\overrightarrow{a}}^{-1}))\text{ at } \mathbf{s}_{\overrightarrow{a}}(\widehat{q})\text{ is non-zero.}
    \end{multline}
    Here the \enquote{only if}-direction follows from the fact that the nondegeneracy of the Jacobian ensures that the linear independence of the spanning vectors is preserved.
    
    For some $\overrightarrow{p}=(p_j)^n_{j=0}\in Y(q)$ with $\aa$ associated we define $\mc{X}_{\overrightarrow{p}}\subset Y(q)$ to be the set of $(\widetilde{p}_j)^n_{j=0}\in Y(q)$, such that for the associated $\overrightarrow{b}$ there exists $W_{\overrightarrow{b}}$ such that $(W_{\overrightarrow{b}},\mathbf{s}_{\overrightarrow{b}})$ are $C^0$-coordinates around $\widehat{q}$ and condition \eqref{eq:compatibility} is satisfied.
    
    If $\overrightarrow{p}\in Y_0(q)$ we see that $Y_0(q) \subset \mc{X}_{\overrightarrow{p}}$. On the other hand $\overrightarrow{p}\notin Y_0(q)$ we have $Y_0(q) \cap \mc{X}_{\overrightarrow{p}} =\emptyset$. Since the set $Y_0(q)$ is open and dense in $Y(q)$, we see that $\overrightarrow{p}\in Y_0(q)$ if and only if the interior of $\mc{X}_{\overrightarrow{p}}$ is dense subset of $Y(q)$. Since the data \ref{rmk:data} is sufficient to determine $Y(q)$ and $\mc{X}_{\overrightarrow{p}}$, we can determine whether $\overrightarrow{p}\in Y_0(q)$ or not. And since, by proposition \ref{prop:observationtimecoordinates}, the $C^0$-observation coordinates $(W_\aa,\mathbf{s}_\aa)$ around $\widehat{q}=\F(q)$ are $C^\infty$-observation coordinates if and only if $\overrightarrow{p}\in Y_0(q)$, where $\overrightarrow{p}$ are associated to $\aa$ wrt. $q$, we can determine all $C^0$-observation coordinates around $\widehat{q}$ which are also $C^\infty$-observation coordinates.
\end{proof}


\section{Construction of the conformal type of the metric}
We will denote by $\widehat{g}=\F_*g$ the metric on $\widehat{V}=\F$ that makes $\F:V\to \widehat{V}$ an isometry. The next lemma will allow us to determine a time-orientation on $\widehat{V}$ making $\F:V\to \widehat{V}$ a causal map, and a metric $G$ which is conformally equivalent to $\widehat{g}$. The key ideas are that we can use some equations from proposition \ref{prop:observationtimecoordinates} to determine a timelike, future-pointing vector field on $\widehat{V}$, and that for a given $\widehat{q}\in \widehat{V}$ we can determine all null geodesics through $\widehat{q}$ allowing us to determine all null cones $L_{\widehat{q}}\widehat{V}$ in $\widehat{V}$.
\begin{lemma}\label{lem:constructmetric}
The data given in \ref{rmk:data} allows us to determine a metric $G$ on $\widehat{V}=\F(V)$ that is conformal to $\widehat{g}$ and a time orientation on $\widehat{V}$ that makes $\F:V\to \widehat{V}$ a causality preserving map.
\end{lemma}
\begin{proof}
    Let $(W_\aa,\mathbf{s}_\aa)$ be $C^\infty$-observation coordinates on $\widehat{V}$ and $\widehat{q}\in W_\aa$.
    We begin by constructing a time orientation on $\widehat{V}$:
    Let $a_1, a_2\in \aa$ and $p_1,p_2\in U$ be associated wrt. the point $q=\F^{-1}(\widehat{q})$, i.e. $p_i=\mc{E}_{a_i}(q)$.
    Because $\mathbf{f}_\aa = \mathbf{s}_\aa \circ \F$ are smooth coordinates we have that the vectors $w(q,p_1)$ and $w(q,p_2)$ pointing from $q$ to $p_i$ must be non-parallel. Therefore, by equation \eqref{eq:fgradient} we see that the gradient vectors $\nabla f_{a_i}(q)$ are non-parallel, lightlike and past-pointing. 
    Thus the co-vectors $-ds_{a_1}\rvert_{\widehat{q}}$ and $-ds_{a_2}\rvert_{\widehat{q}}$ are non-parallel lightlike and future-pointing. This follows from the fact that $\F$ is an isometry and the co-vector $df_a$ is the image of  $\nabla f_a$ under the canonical isomophism. Moreover because the data allows us to fully determine $s_{a_1}$ and $s_{a_2}$ on $\widehat{V}$ (see previous proof) we can also determine $ds_{a_1}$ resp. $ds_{a_2}$.
    
    The co-vector field $X=(-ds_{a_1})+(-ds_{a_2})$ is timelike and future-pointing and forms a local time-orientation on $W_\aa$. Using bump functions and a partition of unity we can then obtain a time-orientation on the whole of $\widehat{V}$ since all orientations agree where they overlap.
    
    Now we turn our attention to the construction of a metric $G$ which is conformal to $\widehat{g}$: Let again $(W_\aa,\mathbf{s}_\aa)$ be $C^\infty$-observation coordinates on $\widehat{V}$ with $\widehat{q}_0\in W_\aa$ and $q_0\in V$ such that $\widehat{q}_0=\F(q_0)$. As in the previous proof, using the data given in $\ref{rmk:data}$ and the function $\widehat{q_0}=F_{q_0}$ we can determine $\ee(q_0), \er(q_0), \mc{D}_K(q_0)$ and $\mc{D}^{reg}_K(q_0)$ by \ref{prop:obsreconstr}.
    
    We then fix the point $\widehat{q_0}=\F(q_0)$ and the tuple $(p,v)\in \mc{D}^{reg}_K(q_0)$. Let $\widehat{t} >0$ be the largest number such that the geodesic $\gamma_{p,v}((-\widehat{t},0])\subset M$ is defined and has no cut point. For $q\in V$, we have that $q\in \gamma_{p,v}((-\widehat{t},0))$ if and only if $(p,v)\in \mc{D}^{reg}_K(q)$. Hence for a fixed $(p,v)\in \mc{D}^{reg}_K(q_0)$ the data allows us to whether some $\widehat{q}\in W_\aa$ has $q=\F^{-1}(\widehat{q}) \in \gamma_{p,v}((-\widehat{t},0))$ by checking if $(p,v)\in \mc{D}^{reg}(q)$. This allows us to determine
        \[
        \beta = \{ \widehat{q}\in W_\aa \mid \widehat{q}=\F(q), \mc{D}^{reg}_K(q) \ni (p,v)\} = \F(\gamma_{p,v}((-\widehat{t},0))) \cap W_\aa.
    \]
    Therefore, on $W_\aa\subset \widehat{V}$ we can find the image, under the map $\F$, of the light-like geodesic segment $\gamma_{p,v}((-\widehat{t},0))\cap \F^{-1}(W_\aa)$ that contains $q_0=\gamma_{p,v}(-t_1)$. Let $\alpha(s), s\in (-s_0,s_0)$ be a smooth path on $W_\aa$ such that $\partial_s\alpha(s)$ is never zero, $\alpha((-s_0,s_0))\subset \beta$ and $\alpha(0) = \widehat{q}_0$. Such a smooth path can, for example be obtained by endowing $\widehat{V}$ with some arbitrary Riemannian metric and parameterizing by arc-length. Then $\widehat{w}=\partial_s\alpha(s)\rvert_{s=0}\in T_{\widehat{q}_0}\widehat{V}$ has the form $\widehat{w}=c\F(\gamma'_{p,v}(-t_1))$ where $c\neq 0$. 
    
    Since we can do the above construction for all points $(p,v)\in \mc{D}^{reg}_U(q_0)$, we can determine in the tangent space $T_{\widehat{q}_0}\widehat{V}$ the set 
    \[
    \Gamma = \F_*(\{cw\in L_{q_0}M \mid \exp_{q_0}(w) \in \er(q_0), c\in \R\setminus \{0\}\}) 
    \]
    which is an open, non-empty subset of the light cone at $\widehat{q}_0$ wrt. the metric $\widehat{g}$. But now, since the light cone is determined by a quadratic equation in the tangent space, having an open set $\Gamma$ determines the whole light cone. By repeating this construction for all points $\widehat{q}\in \widehat{V}$, we can uniquely determine $L\widehat{V}$. Using proposition \ref{prop:metricfromnullcone} we can then determine the conformal class of the tensor $\widehat{g}=\F_*g$ in the manifold $\widehat{V}$.
    
    The above shows that the data \ref{rmk:data} determine the conformal class of the metric tensor $\widehat{g}$. And in particular we can construct a metric $G$ on $\widehat{V}$ that is conformal to $\widehat{g}$ and satisfies $G(X,X)=-1$.
\end{proof}

\section{Reconstruction overview}
We have gone through all the steps necessary to reconstruct the conformal, differential and topological data of $V$ and will now tie this all together to give a detailed account of the actual reconstruction.

As mentioned in remark \ref{rmk:data} we want to prove the following theorem which implies theorem \ref{thm:intreconstr}:
\begin{theorem}
    Let $(M,g)$ be a globally hyperbolic Lorentzian manifold and $p^+,p^-\in M, V\subset \jp$ suitable such that $V$ is an open subset of $\interior{\jp}$. Then given 
    \begin{enumerate}[label={\textnormal{(\arabic*)}}]
        \item The smooth manifold $K$,
        \item the conformal class of $g\rvert_K$ and
        \item the set of light cone observations $\mc{P}_K(V)$
    \end{enumerate}
    we can construct a globally hyperbolic Lorentzian manifold $\widehat{V}$ such that there exists a conformal diffeomorphism $\F:V\to \widehat{V}$, which preserves causality.
\end{theorem}
\begin{proof}
    % Assume we are given data $\ref{rmk:data}$, that is, the smooth manifold $U$, the conformal class of $g\rvert_U$, the family of smooth and timelike future-pointing paths $\mu_a, a\in \ca$ and the collection of light observation sets $\mc{P}_U(V)=\{\mc{P}_U(q)\mid q\in V\}$. Note that for a given  $\mc{P}_U(q)$, we cannot, a priory, determine to which $q\in V$ it belongs.
    
    To construct the space $\widehat{V}$ which is conformally diffeomorphic to $V$ we follow these steps:
    \begin{itemize}
        \item As $f_a(q) = \min \{s\in [-T_a,0] \mid \mu_a(s) \in \mc{P}_K(q)\}$ we can determine $\mc{E}_K(V) = \{\ee(q) \mid q\in V\}$ from $\mc{P}_K(V)$.
        \item Proposition \ref{prop:obsreconstr} then allows us to determine $\mc{D}^{reg}_K(q)$, $\mc{D}_K(q)$ and $\er(q)$ for a given $\ee(q)\in \ee(V)$. We can thus construct $\mc{D}^{reg}_K(V)$, $\mc{D}_K(V)$ and $\er(V)$.
        \item We define the function 
        \begin{align*}
            \F:V&\to \F(V) = \widehat{V} \subset (\mc{C}^\infty(\sn),d)\\
            q &\mapsto \widehat{q} = F_q = (a\mapsto f_a(q)).
        \end{align*}
        For a given $\ee(q)$ we can construct $\widehat{q}$ by $\widehat{q}(a) = f_a(q) = s$ such that $\mu_a(s) \in \ee(q)$.
        This allows us to construct the map
        \begin{align*}
            \widetilde{\F}:\ee(V)&\to \widehat{V}\\
            \ee(q) &\mapsto \widehat{q}.
        \end{align*}
        And we can thus determine the set $\widehat{V} = \widetilde{\F}(\ee(V))$.

        \item By taking the subspace topology with respect to the topology on $\mc{C}^\infty(\sn)$ induced by $d$ we can determine a topology on $\widehat{V}$. By corollary \ref{cor:Fhomeo} this topology is homeomorphic to the topology on $V$, making $\F$ a homeomorphism.
        
    %     Note that by the same procedure we can define the map
    %     \begin{align*}
    %         \widetilde{f}_a:\varepsilon_U(V)&\to\R\\
    %         \varepsilon_U(q) &\mapsto f_a(q).
    %     \end{align*}
        
    %     Furthermore if we endow $\widehat{V}$ with the product topology and let $\pi_a:\widehat{V}\to \R$ with $\widehat{q}\mapsto \widehat{q}(a)$ then the following diagrams commute for all $a\in \ca$:
    %     \[\begin{tikzcd}
    %     V &&& {\widehat{V}} \\
    %     {\varepsilon_U(V)} & {\widehat{V}} & V & {\mathbb{R}} & {\varepsilon_U(V)}
    %     \arrow["{\mathcal{F}}", from=1-1, to=2-2]
    %     \arrow["{\widetilde{\mathcal{F}}}"', from=2-1, to=2-2]
    %     \arrow["{q\mapsto\varepsilon_U(q)}"', from=1-1, to=2-1]
    %     \arrow["{f_a}"', from=2-3, to=2-4]
    %     \arrow["{\widetilde{f}_a}", from=2-5, to=2-4]
    %     \arrow["{\mathcal{F}}", from=2-3, to=1-4]
    %     \arrow["{\widetilde{\mathcal{F}}}"', from=2-5, to=1-4]
    %     \arrow["{\pi_a}", from=1-4, to=2-4]
    % \end{tikzcd}\]
    
    % Note that in the above diagrams, the data \ref{rmk:data} only allows us to construct the sets $\varepsilon_U(V)$ and $\widehat{V}$ and the maps $\widetilde{F}$, $\widetilde{f}_a$ and $\pi_a$.
    
    % We thus can construct the space $\widehat{V}$ and endow it with the product topology since both $\ca$ and $\R$ are metric spaces.
    
    \item For a given point $\widehat{q}\in \widehat{V}$ we can use proposition \ref{prop:findsmoothcoords} and the data to determine all $C^0$-observation coordinates around $\widehat{q}$. We can then determine for each of these coordinates if they are also $C^\infty$-observation coordinates, and find at least one such coordinate system since existence is guaranteed. We can repeat that step for each $\widehat{q}\in \widehat{V}$ to find smooth coordinates on $\widehat{V}$, making $\F$ a diffeomorphism. 
    
    \item Finally we can use lemma \ref{lem:constructmetric} to construct a metric $G$ and time-orientation $X$ on $\widehat{V}$ which is conformal to $\widehat{g}=\F_*g$ and makes $\F$ causal. 
    $\F:(V,g\rvert_U)\to (\widehat{V},G)$ is thus a causal conformal diffeomorphism as desired.
    \end{itemize}
\end{proof}

\begin{remark}\label{rmk:ReconstrPastBoundary}
    In the statement of theorem \ref{thm:intreconstr} we required that $V$ be a subset of the interior $\interior{\jp}$. This is because as we approach the observation set $K$, the light cone observation sets get increasingly degenerate and loose many of their nice properties for points on the boundary, i.e. if we had a $q\in V\cap K$. This issue will be adressed in the next chapter by smoothing the observation time functions at the boundary.

    However if $q\in V$ approaches the past boundary $K^-\coloneqq \mc{L}^+_{p^-}\cap I^-(p^+)$ the situation is simpler: Because we are always away from the set of observers $K$, the light cone observation sets remain transverse to $K$ and thus well behaved even for $q\in \mc{L}^+_{p^-}\cap I^-(p^+)$. It is thus possible to relax the condition $V\subset \interior{\jp}$ to $V\subset \jp \setminus K$ in theorem \ref{thm:intreconstr} with only minor modifications to the proofs.
\end{remark}