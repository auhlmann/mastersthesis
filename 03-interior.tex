\chapter{Interior Reconstruction}

\section{Construction of the topology}
We aim to reconstruct the topological and differential data of $V$. To that end we define the following functions.

For $q\in V$ we define the function $F_q:\sn\to\R$ by $a\mapsto f_a(q)$. We can then define the function 
\begin{align*}
    \mathcal{F}:V &\to (C(\sn),d_{\infty})\\
    q&\mapsto F_q
\end{align*} mapping a $q\in V$ to the function $F_q:\sn\to\R$. $(C(\sn),d_{\infty})$ is the space of continuous functions from $\sn$ to $\R$, together with the metric $d_\infty(f,g)=\max_{a\in \sn} \rvert f(a)-g(a) \lvert$.

The following proposition establishes that the canonical topological structure on $\mc{F}(V)$, i.e. the topology obtained by taking the subspace topology wrt. the topology induced by $d_\infty$ on $C(\sn)$, is the same as the pushforward under $\mc{F}$ of the topology on $V$, making $\mc{F}$ a homeomorphism.
((Explain that data allows us to determine $d_\infty$ ))
\begin{lemma}\label{lem:Fprop}
The map $\mathcal{F}:V\to\widehat{V}:=\mathcal{F}(V)$ is a well-defined continuous and injective.
\end{lemma}
\begin{proof}
First of all $\F:V\to (C(\sn),d_\infty)$ is well-defined by proposition \ref{prop:fcont}, i.e. for any $q\in V$, $F_q=f(q,\cdot)$ is continuous on $\sn$.

Because the topology induced on $C(\sn)$ by $d_\infty$ is uniform convergence, proposition \ref{prop:funif} implies that $\mc{F}$ is continuous.

Finally injectivity follows from the fact that for any $q,q'\in V$ we have $\F(q)=\F(q')\implies F_q = F_{q'} \implies \ee(q) = \ee(q')$ which implies $q = q'$ by proposition \ref{prop:einj}.
\end{proof}

However there is still some work required to show that also $\F^{-1}$ is continous on $\widehat{V}$:
\begin{proposition}\label{prop:frevcont} Let 
    $(q_n)_{n=1}^{\infty}$ be a sequence in $V$ and $q_0\in V$ such that $F_{q_n}\to F_{q_0}$ uniformly then $q_n \to q_0$ as $n\to \infty$.
\end{proposition}
\begin{proof}
    Let $(q_n)_{n=1}^\infty, q_0 \in V$ such that $F_{q_n}\to F_{q_0}$ uniformly. Note that because $\F$ is injective the choice of $q_n$ for $F_{q_n}=\F(q_n)$ is unambigous.

    Let $t_0 = \max_{a\in \sn} F_{q_0}(a)$, because $q_0\in V$, lemma \ref{lem:Kcharact}(3) implies that $t_0<1$. Hence there exists an $\varepsilon>0$ such that $t':=t_0 + \varepsilon < 1$. And because $F_{q_n}$ to $F_{q_0}$ uniformly there exists a $N\in \N$ such that $t_n:=\max_{a\in \sn} F_{q_n}(a)<t'$ for all $n \ge N$. 

    We now define 
    \[
        C_{t'}:=\{q\in \jp \mid \max_{a\in \sn} F_q(q)\leq t'\}   
    \] and after removing the first $N$ elements of $(q_n)_{n=1}^\infty$ we may assume $q_n, q_0\in C_{t'}$.

    We now show that $C_{t'}\subset \jp$ is closed: Let $p_n \to p_0 \in \jp$ with $p_n \in C_{t'}$. We must have $p_0\in \interior{\jp}$ because otherwise $\max_{a\in \sn} F_{p_n}(a)\to 1 > t'$. Because $p_n, p_0$ does not necessarily lie in $V$ we but we still want to use all the machinery we built up so far we now look at $V_0:= \interior{\jp}$. We can do this because all results were stated in terms of any arbitrary $V\subset \interior{\jp}$ which includes the case $V_0 = \interior{\jp}$. We then apply proposition \ref{prop:funif} to $V_0$, to get $F_{p_n} \to F_{p_0}$ uniformly. This in turn implies $\max_{a\in \sn}F_{p_n} \to \max_{a\in \sn}F_{p_0}$ and because $p_n\in C_{t'}$ we have $\max_{a\in \sn}F_{p_n}\leq t'$ for all $n$ and hence also $\max_{a\in \sn}F_{p_0}\leq t'$. This proves $p_0\in C_{t'}$ making $C_{t'}$ closed.

    Now $C_{t'}$ is a closed subset of a compact space and thus itself compact. Again using $V_0$ and the previous lemma we get that $\F:V_0 \to \F(V_0)$ is well-defined continuous and injective. We can then restrict $\F$ to $C_{t'}$ and all these properties are preserved. But now $\F:C_{t'}\to \F(C_{t'})$ is a continuous, injective map from a compact space to a hausdorff space, making it a homeomorphism. 
    
    Because $q_n,q_0\in C_{t'}$ for all $n$, we have $F_{q_n}=\F(q_n)\to F_{q_0} = \F(q_0)\in \F(C_{t'})$. Using that $\F^{-1}$ continuous on $\F(C_{t'})$ we get
    \[
        q_n = \mc{F}^{-1}(F_{q_n}) \to \mc{F}^{-1}(F_{q_0}) = q_0,
    \] as desired.
\end{proof}

And we get:
\begin{corollary}\label{cor:Fhomeo}
    $\F:V\to \widehat{V}$ is a homeomorphism.
\end{corollary}


\section{Smooth Reconstruction}
Having established the topological structure of $V$ we next aim to establish coordinates on $\F(V)$ near any $\F(q)$ that make $\F(V)$ diffeomorphic to $V$.

\subsection{Preparation}

\begin{definition}[Coordinates on $V$]
We first define 
\[
    \mc{Z} = \{(q,p)\in V\times K \mid p\in \mc{E}_K^{reg}(q)\}.
\] 
Then for every $(q,p)\in \mc{Z}$ there is a unique $w\in L^K_qM$ such that $\gamma_{q,w}(1)=p$ and $\rho(q,w)>1$. Existence follows from lemma \ref{lem:observationtime} while uniqueness follows from the fact that $p\in \mc{E}_K^{reg}(q)$ and thus cannot be a cut point. 
We can then define the map
\begin{align*}
    \Omega:\mc{Z}&\mapsto L^KV\\
    (q,p)&\mapsto (q,w)
\end{align*}
Note that this map is injective.
Below we will $\mc{W}_\varepsilon(q_0,w_0)\subset TM$ be a $\varepsilon$-neighborhood of $(q_0,w_0)$ with respect to the Sasaki-metric induced on $TM$ by $g^+$.
\end{definition}

\begin{lemma}\label{lem:Tprop}
    ((Move to appendix?))The function
    \begin{align*}
        T_+:L^+\jp&\to \R \\
        (q,w)&\mapsto \sup \{ t\ge 0 \mid \gamma_{q,w}(t)\in J^-(p^+)\}
    \end{align*}
    is finite and upper semicontinuous.
\end{lemma}
\begin{proof}
    Finiteness follows from lemma \ref{lem:leavescompact}. We now want to show that $T_+$ is upper semicontinuous. To that end let $(q_n,w_n)\to (q_0,w_0)\in L^+\jp$, we want to show that $\limsup_{n\to \infty} T_+(q_n,w_n) \leq T_+(q_0,w_0)$: Let $\varepsilon>0$ and set $t_0 = T_+(q_0,w_0)$. Then by definition we have $\gamma_{q_0,w_0}(t_0)\in M \setminus J^-(p^+)$. Because $\gamma_{q_n,w_n}(t_0) \to \gamma_{q_0,w_0}(t_0)$ and $M \setminus J^-(p^+)$ open, there exists a $N\in \N$ such that $\gamma_{q_n,w_n}(t_0)\in M \setminus J^-(p^+)$ for all $n\ge N$. Note that if $\gamma_{q_n,w_n}(t_0)\notin J^-(p^+)$ then for any $t'\ge t_0$ we also have $\gamma_{q_n,w_n}(t') \notin J^-(p^+)$ because otherwise we could obtain a lightlike path from $\gamma_{q_n,w_n}(t_0)$ to $p^+$, a contradiction. Thus, by definition $T_+(q_n,w_n)\leq t_0$ and 
    $\limsup_{n\to \infty} T_+(q_n,w_n) \leq t_0 = T_+(q_0,w_0)+\varepsilon$. Finally because $\varepsilon>0$ was arbitrary we get $\limsup_{n\to \infty} T_+(q_n,w_n) \leq T_+(q_0,w_0)$ as desired.
\end{proof}

\begin{lemma}
Let $(q_0,p_0)\in \mc{Z}$ and $(q_0,w_0)=\Omega(q_0,p_0)$. When $\varepsilon>0$ is small enough the map 
\begin{align*}
    X:\mc{W}_\varepsilon(q_0,w_0) &\to M\times M\\
    (q,w) &\mapsto (q,\exp_q(w))
\end{align*}
is open and defines a diffeomorphism $X:\mc{W}_\varepsilon(q_0,w_0)\to \U_\varepsilon(q_0,p_0) \coloneqq X(\mc{W}_\varepsilon(q_0,w_0))$. When $\varepsilon$ is small enough, $\Omega$ coincides in $\mc{Z}\cap \U_\varepsilon(q_0,p_0)$ with the inverse map of $X$. Moreover $\mc{Z}$ is a $2n$-dimensional manifold and the map $\Omega:\mc{Z}\to L^KM$ is smooth.
\end{lemma}

\begin{proof}
    Because $p_0\in \er(q)$ we have $\rho(q_0,w_0)>1$. Because $\rho$ is lower semicontinuous, for $\varepsilon>0$ small enough we have $\rho(q',w')>$ for all $(q',w')\in \mc{W}_{\varepsilon}(q_0,w_0)\subset TV$. Thus $X:\mc{W}_{\varepsilon}(q_0,w_0)\to \mc{U}_{\varepsilon}(q_0,p_0) = X(\mc{W}_{\varepsilon}(q_0,w_0))$ is a diffeomorphism with $\mc{U}_{\varepsilon}(q_0,p_0)$ open in $M\times M$ by the invariance of of domain theorem. 
    
    Next we aim to show that $\Omega:\mc{Z}\to L^KV$ is continuous at $(q_0,p_0)\in \mc{Z}$. We proceed by assuming there exists a sequence $(q_n,p_n)\in \mc{Z}$ converging to $(q_0,p_0)$ such that $\Theta(q_n,p_n)=(q_n,w_n)\in L^+V$ does not converge to $\Theta(q_0,p_0)=(q_0,w_0)$.

    First of all we aim to show that the sequence $(q_n,w_n)$ is bounded and thus has a convergent subsequence: Because $q_n\to q_0$ we only need to show that $w_n$ is bounded. To that end we introduce an arbitrary riemannian metric consistent with the topology on $M$ and can write $w_n = t_n \overline{w_n}$ where $\lVert \overline{w_n} \rVert_{g^+}=1$. To show that $t_n$ is bounded we first define 
    \[
        C:=\{(q,w)\in L^+M \mid q\in \jp \text{ and } \lVert w \rVert_{g^+}=1\}
    \]
    and $C$ is compact and because $T_+$ is upper semicontinuous on $C$, there exists a $c_0>0$ such that $T_+(q,w)\leq c_0$ for all $(q,w)\in C$.
    Recall that we have $\gamma_{q_n,\overline{w_n}}(t_n)=\exp_{q_n}(w_n)=p_n \in K \subset \jp$. Together with $(q_n,\overline{w_n}) \in C$ this yields 
    \[
        \lVert w_n \rVert_{g^+} = t_n\lVert \overline{w_n} \rVert_{g^+} = t_n \leq T_{+}(q_n,\overline{w_n})<c_0,
    \] proving $(q_n,w_n) \in L^KV$ is bounded.

    We can thus obtain a convergent subsequence $(q_k,w_k)=\Theta(q_k,p_k)\to (q_0,w')$ with $w'\neq w_0$. Since the exponential map is continuous, we would have 
    \[
        \exp_{q_n}(w') = \lim_{n\to \infty} \exp_{q_n}(w_n) = \lim_{n\to\infty} p_n = p_0 = \exp_{q_n}(w_0).
    \] with $w'\neq w_0$. But since $p_0\in \er(q)$ cannot be a cut point this is a contradiction and $\Omega:\mc{Z}\to L^KV$ must be continuous.
    
    Next we use the fact that $\Omega$ is continuous and get $\Omega^{-1}(\mc{W}_{\varepsilon}(q_0,w_0)) \subset \mc{Z}$ is open. We can thus find a $\varepsilon_1\in (0,\varepsilon)$ such that for the open ball $\mc{U}_{\varepsilon_1}(q_0,w_0)\subset M$ we have 
    \[
        \mc{Y}_{\varepsilon_1}:=\mc{U}_{\varepsilon_1}(q_0,w_0) \cap \mc{Z} \subset \Omega^{-1}(\mc{W}_{\varepsilon}(q_0,w_0))
    \] implying $\Omega(\mc{Y}_{\varepsilon_1}) \subset \mc{W}_{\varepsilon}(q_0,w_0)$. Then for $(q,p)\in \mc{Y}_{\varepsilon_1}$ and $(q,w) = \Omega(q,p)\in \mc{W}_\varepsilon(p_0,w_0)$ we have $\exp_q(w) = p$. Hence $X(\Omega(q,p))=(q,p)$. But now since $(q,p)\in \U_\varepsilon(p_0,q_0)$ we can apply $X^{-1}$ to both sides and get $\Omega(q,p)=X^{-1}(q,p)$. Thus on $\mc{Y}_{\varepsilon_1}$ the function $\Omega:\mc{Y}_{\varepsilon_1}\to TM$ coincides with the smooth function $X^{-1}:\mc{Y}_{\varepsilon_1}\to TM$, which implies that $\Omega$ is smooth with full rank differential on $\mc{Y}_{\varepsilon_1}$ as well.
    
    Now since $(q_0,p_0)\in \mc{Z}$ was arbitrary we get that $\Theta:\mc{Z}\to L^+V$ is smooth everywhere, injective and locally diffeomorphic with full rank. Thus $\mc{Z}$ diffeomorphic to an open subset of $L^KV$. This makes it a manifold with dimension $(n+1)+(n-1)=2n$.
\end{proof}

\begin{proposition}\label{prop:observationtimecoordinates}
Let $q\in V$ and $(q_0,p_j)\in \mc{Z}, j=0,\dots, n$ and $w_j\in L^K_{q_0}M$ such that $\gamma_{q_0,w_j}(1) = p_j$. Assume that $w_j, j=1,\dots, n$ are linearly independent. Then, if $a_j\in A$ and $\overrightarrow{a} = (a_j)^n_{j=1}$ are such that $p_j\in \mu_{a_j}$, there is a neighborhood $V_1\subset M$ of $q_0$ such that the corresponding observation time functions 
\[
\mathbf{f}_{\overrightarrow{a}}(q) = (f_{a_j}(q))^n_{j=0}
\]
define smooth coordinates on $V_1$. Moreover $\nabla f_{a_j}\rvert_{q_0}$, i.e. gradient of $f_{a_j}$ with respect to $q$ at $q_0$, satisfies $\nabla f_{a_j}\rvert_{q_0} = c_jw_j$ for some $c_j\neq 0$.
\end{proposition}

\begin{proof}
    First we need some setup: Let $(q_0,p_0)\in \mc{Z}$ and $w_0\in L^+_{q_0}M$ such that $\gamma_{q_0,w_0}(1) = p_0$. Furthermore let $\varepsilon>0$ be small enough such that the map $X:\mc{W}_\varepsilon(q_0,w_0)\to \U_\varepsilon(q_0,p_0)$ is a diffeomorphism (see the previous lemma). We will denote this inverse by $X^{-1}(q,p) = (q,w(q,p))$ and write $\mc{W}=\mc{W}_\varepsilon(q_0,w_0), \U=\U_\varepsilon(q_0,p_0)$.
    
    We associate with any $(q,p)\in \mc{U}$ the energy $E(q,p)=E(\gamma_{q,w(q,p)}([0,1]))$ of the geodesic segment connecting $q$ to $p$. The energy of a piecewise smooth curve $\alpha:[0,l]\to M$ is defined as 
    \[
        E(\alpha) = \frac{1}{2}\int_0^l g(\alpha'(t),\alpha'(t))dt.
    \]
    Note that the sign of $E(\alpha)$ depends on the causal nature of $\gamma_{q,w(q,p)}$. In particular $E(q,p)=0$ if and only if $w(q,p)$ is light-like. Moreover, as $X^{-1}$ is smooth on $\U$, so is $E(p,q)$.
    
    We now return to consider $(q_0,p_0)\in \mc{Z}$ and let $a\in \sn$ be such that $p_0 \in \mu_a$. Then $p_0=\mu_a(s_0)$ with $s_0=f_a(q_0)$ as $p_0 \in \er(q_0)$ and $s_0\in (0,1)$ by lemma \ref{lem:observationtime}(1). 
    
    Let $V_0\subset V$ be an open neighborhood of $q_0$ and $t_1,t_2\in (0,1), t_1<s_0<t_2$, such that $V_0\times\mu_a([t_1,t_2])\subset \U$, which exist because $\U$ is open. Then for any $q\in V_0, s\in (t_1,t_2)$ the function $\mathbf{E}_a(q,s)\coloneqq E(q,\mu_a(s))$ is well defined and smooth.
    
    We want to use first variation formula for $\mathbf{E}_a(q,s)$ ((E Reference)) to calculate $\left.\frac{\partial\mathbf{E}_a(q_0,s)}{\partial s}\right\rvert_{s=s_0}$ and $\left.\nabla_q\mathbf{E}_a(q,s_0)\right\rvert_{q=q_0}$.
    
    For the first part we define the variation $\x(t,s) = \gamma_{q_0,w(s)}(t), t\in [0,1]$ where $w(s):= w(q_0,\mu_a(s+s_0)), s\in [t_1-s_0,t_2-s_0]$. Note that $\x(t,0)=\gamma_{q_0,w_0}(t)$. We then get
    \begin{align*}
        \left.\frac{\partial\mathbf{E}_a(q_0,s)}{\partial s}\right\rvert_{s=s_0} = E'_{\x}(0) =  \left.g(V,\gamma_{q_0,w_0}')\right\rvert^1_0
    \end{align*}
    since $\gamma_{q_0,w_0}$ is a geodesic and $\x$ has no breaks. If we now further notice that $V(0)=0$ as $\x(0,s)=q_0$ for all $s\in[t_1,t_2]$ and $V(1) = \mu_a'(s_0) = \mu_a'(f_a(q_0))$ as $\x(1,s) = \mu_a(s+s_0)$ we can conclude 
    \begin{align*}
        \left.\frac{\partial\mathbf{E}_a(q_0,s)}{\partial s}\right\rvert_{s=s_0} &= g(V(1),\gamma'_{q_0,w_0}(1)) - g(V(0),\gamma'_{q_0,w_0}(0))\\
        &= g(\mu_a'(f_a(q_0)),\gamma'_{q_0,w_0}(1))
    \end{align*}
    
    For the second part we will introduce coordinates $\mathbf{q}=(q_0,\dots,q_n)$ around $q_0$. Then the gradient can be written as
    \[
        \left.\nabla_q\mathbf{E}_a(q,s_0)\right\rvert_{q=q_0} = g^{ij}\left.\frac{\partial\mathbf{E}_a(q,s_0)}{\partial q_i}\right\rvert_{q=q_0}\partial_j.
    \]
    To calculate $\left.\frac{\partial\mathbf{E}_a(q,s_0)}{\partial q_i}\right\rvert_{q=q_0}$ we now introduce variations $\x_i(t,s) = \gamma_{q(s),w(s)}(t)$ where $w(s):=w(q(s),\mu_a(s_0))$ and $q(s):=q^{-1}(q_0(q_0),\dots, q_i(q_0)+s, \dots q_n(q_0))$ is obtained by increasing the $i$-th coordinate by $s$. Note that these variations all have $\x_i(t,0) = \gamma_{q_0,w_0}(t)$, $\x_i(1,s) = \mu_a(s_0)$ thus $V_{\x_i}(1)=0$ and $V_{\x_i}(0) = \frac{\partial}{\partial s}\x_i(0,s)\rvert_{s=0} = \partial_i$.
    After again applying proposition ((E REF))
    \[
        \left.\frac{\partial\mathbf{E}_a(q,s_0)}{\partial q_i}\right\rvert_{q=q_0} = E'_{\x_i}(0) =  -g(V(0),\gamma'_{q_0,w_0}(0)) = -g(\partial_i,w_0).
    \]
    Combining this with coordinate representation of the gradient we get
    \begin{align*}
        \left.\nabla_q\mathbf{E}_a(q,s_0)\right\rvert_{q=q_0} &= g^{ij}\left.\frac{\partial\mathbf{E}_a(q,s_0)}{\partial q_i}\right\rvert_{q=q_0}\partial_j = -g^{ij}(g_{\alpha\beta}\partial_i^\alpha w_0^\beta)\partial_j\\
        &= -g^{ij}g_{i\beta}w_0^\beta\partial_j = -\delta^j_\beta w_0^\beta \partial_j\\
        &= -w_0^j\partial_j = -w_0.
    \end{align*}
    
    We thus managed to calculate what we wanted and can summarize as
    \begin{equation}\label{eq:dEsummary}
        \left.\frac{\partial\mathbf{E}_a(q_0,s)}{\partial s}\right\rvert_{s=s_0} = g(v,\mu'_a(f_a(q_0))), \quad \left.\nabla_q\mathbf{E}_a(q,s_0)\right\rvert_{q=q_0}=-w_0
    \end{equation}
    where $w_0=w(q_0,p_0)$ and $v=\gamma'_{q_0,w_0}(1)$. Since $\mu'_a(f_a(q_0))$ and $v$ are both future-pointing null vectors, which by lemma \ref{lem:transversality} must be transversal we have $\left.\frac{\partial\mathbf{E}_a(q_0,s)}{\partial s}\right\rvert_{s=s_0}=g(v,\mu_a'(f_a(q)))<0$.

    We can now use the implicit function theorem on $V_0\times [t_1,t_2]$ with equation $E_a(q,s)=0$ and single solution $E_a(q_0,s_0)=0$. This yields an open neighborhood $V_a\subset V_0$ and a smooth function $q\mapsto s_a(q)$ such that $E_a(q,s_a(q)) = 0$ for all $q \in V_a$. Now $E_a(q,s_a(q)) = E(q,\mu_a(s_a(q)))=0$, implies $\mu_a(s_a(q))\in \pq$. This together with $(q,s_a(q))\in \U$ implies that $\mu_a(s_a(q))\in \er(q)$ and thus $s_a(q) = f_a(q)$ on $V_a$. Hence we have $\left.\nabla f_a(q)\right\rvert_{q=q_0} = \left.\nabla s_a(q)\right\rvert_{q=q_0}$ and from equation \ref{eq:dEsummary} together with the implicit function theorem it follows that 
    \begin{equation}\label{eq:fgradient}
        \left.\nabla f_a(q)\right\rvert_{q=q_0}  = \frac{1}{c(q_0,a)}w_0, \quad 
        c(q_0,a) = \left.\frac{\partial\mathbf{E}_a(q_0,s)}{\partial s}\right\rvert_{s=s_0} < 0,
    \end{equation}
    where $p_0 = \mu_a(s_0)=\mc{E}_a(q_0), s_0 = f_a(q_0)$ and $w_0=w(q_0,p_0)$.
    
    Next we choose $p_0, \dots, p_n \in \er(q_0)$ and let $w_0,\dots,w_n\in L^K_{q_0}M$ such that $p_i = \gamma_{q_0,w_i}(1)$, i.e. $w_i=w(q_0,p_i)$. We assume that $w_0,\dots,w_n$ are linearly independent. Moreover let $a_j \in \sn$ such that $p_i\in \mu_{a_j}$ and $\overrightarrow{a}=(a_j)^n_{j=1}$.
    Finally we denote by $q\mapsto s_{a_j}(q)=f_{a_j}(q)$ the above constructed smooth functions which are defined on some neighborhoods $V_{a_j}\subset V$ of $q_0$.
    
    Let $V_{\overrightarrow{a}} = \bigcap_{j=1}^n V_{a_j}$ and consider the map
    \begin{align*}
        \mathbf{f}_{\overrightarrow{a}}:V_{\overrightarrow{a}}&\to \R^n\\
        q&\mapsto (f_{a_1}(q),\dots,f_{a_n}(q)).
    \end{align*}
    Because all of its components are smooth, $\mathbf{f}_{\overrightarrow{a}}$ itself is smooth as well. By equation \ref{eq:fgradient} each component has gradient $\left.\nabla f_{a_j}(q)\right\rvert_{q=q_0} = \frac{1}{c(q_0,a_j)}w_i$ with $c(q_0,a_j)\neq 0$. Since we assumed that $w_0,\dots,w_n$ be independent, $\mathbf{f}_{\overrightarrow{a}}$ is non-degenerate at $q_0$ and thus defines a smooth coordinate system in some neighborhood $V_1$  of $q_0$.
\end{proof}

\subsection{Construction of smooth coordinates}
We will consider $\mc{F}(V)$ a topological space and denote $\mc{F}(V)=\widehat{V}$. We denote the points of this manifold by $\widehat{q}=\mc{F}(q)$. Next we construct a differentiable structure on $\widehat{V}$ that is compatible with that of $V$ and makes $\F$ a diffeomorphism.

\begin{definition}[Observation Coordinates]
Let $\widehat{q}=\F(q)\in \widehat{V}$ and $\aa = (a_j)^n_{j=0}\subset (\sn)^{n+1}$ with $p_j = \mc{E}_{a_j}(q)$ such that $p_j\in \er(q)$ for all $j=0,\dots,n$. Let $s_{a_j} = f_{a_j} \circ \mathcal{F}^{-1}$ and $\mathbf{s}_{\overrightarrow{a}}=\mathbf{f}_{\overrightarrow{a}}\circ\mc{F}^{-1}$. Let $W\subset\widehat{V}$ be an open neighborhood of $\widehat{q}$. We say that $(W,\mathbf{s}_\aa)$ are $C^0$-observation coordinates around $\widehat{q}$ if the map $\mathbf{s}_\aa:W\to \R^n$ is open and injective. Also we say that $(W,\mathbf{s}_\aa)$ are $C^\infty$-observation coordinates around $\widehat{q}$ if $\mathbf{s}_{\overrightarrow{a}}\circ\mc{F}:\mc{F}^{-1}(W)\to \R^n$ are smooth local coordinates on $V\subset M$.
\end{definition}
Note that by the invariance of domain theorem, $\mathbf{s}_{\overrightarrow{a}}:W\to \R^n$ is open if it is injective.
Although for a given $\overrightarrow{a}\in (\sn)^{n+1}$ there might be several sets $W$ for which $(W,\mathbf{s}_{\overrightarrow{a}})$ form $C^0$-observation coordinates to clarify the notation we will often denote the coordinates $(W,\mathbf{s}_{\overrightarrow{a}})$ as $(W_{\overrightarrow{a}},\mathbf{s}_{\overrightarrow{a}})$. 

\begin{proposition}\label{prop:findsmoothcoords}
Let  $\widehat{q}\in \widehat{V}$ then the following holds:
\begin{enumerate}[label={\textnormal{(\arabic*)}}]
    \item Given the data from \ref{rmk:data} we can determine all $C^0$-observation coordinates around $\widehat{q}$,
    \item there exist $C^\infty$-observation coordinates $(W_\aa,\mathbf{s}_\aa)$ around $\widehat{q}$ and
    \item given any $C^0$-observation coordinates $(W_\aa,\mathbf{s}_\aa)$ around $\widehat{q}$, the data \ref{rmk:data}, allows us to determine whether they are $C^\infty$-observation coordinates around $\widehat{q}$.
\end{enumerate}
\end{proposition}

\begin{proof}
    We begin with some setup: Let $q\in V$. We say that $p\in \er(q)$ and $a\in \sn$ are \emph{associated} with respect to $q$ if $p\in \mu_a$, i.e. $p=\mc{E}_a(q)$. 
    
    To prove part (1), we let $\widehat{q}\in \widehat{V}$ with $\widehat{q}=\F(q)$. We want to show that for any choice of observers $\aa=(a_j)^n_{j=0}\in (\sn)^{n+1}$ we can determine if they form $C^0$-observation coordinates. First of all we need to check whether the associated $p_j=\mc{E}_{a_j}(q)$ are regular points, i.e. $p_j\in \er(q)$. But as $\widehat{q}=\F(q) = F_q$ we can recover $\ee{q}=\bigcup_{a\in \sn}\mu_a(F_q(a))$ and also the associated points $p_j = \mu_{a_j}(F_q(a_j))$. By proposition \ref{prop:obsreconstr} this allows us to determine $\er(q)$ and for all $p_j$ we can then simply check whether they lie in $\er(q)$.
    
    We now need to check whether there exists an open neighborhood $W$ of $\widehat{q}$ such that the map $\mathbf{s}_\aa:W\to\R^n$ is injective. 
    By definition we have 
    \[
    \mathbf{s}_\aa(\widehat{q}) = (\widehat{q}(a_1),\dots,\widehat{q}(a_n)) = (F_q(a_0), \dots ,F_q(a_n))
    \] which means that the data allows us to fully determine $\mathbf{s}_\aa$ on $\widehat{V}$. But since by corollary \ref{cor:Fhomeo}, the data allows us do construct the topology on $\widehat{V}$ we can determine whether there exists an open neighborhood $W$ of $\widehat{q}$ such that $\mathbf{s}_\aa:W\to \R^n$ is injective and thus open by the invariance of domain theorem.
    

    To show (2) we let again $\widehat{q}\in \widehat{V}$ with $\widehat{q}=\F(q)$. 
    Let $(a_j)_{j=0}^n \in (\sn)^{n+1}$ such that the associated $p_j \in \er(q)$ and the vectors $\{w_j = w(q,p_j)\mid j=0,\dots n\}$ are linearly independent. We can find such a set of linearly independent vectors because by proposition \ref{prop:submanifolds} $\er(q)$ is an open subset of $\ee(q)$. Now by proposition \ref{prop:observationtimecoordinates} the observation time functions $\mathbf{f}_\aa$ define smooth coordinates on a neighborhood $V_1$ of $q$. Thus $\mathbf{s}_\aa\circ\F$ are smooth local coordinates as well making $(\mathbf{s}_\aa,\F(V_1))$ $C^\infty$-observation coordinates.

    Moving on to part (3): We begin by proving that the set of points in $(\er(q))^{n+1}$ which yield $C^\infty$-observation coordinates is open and dense in $(\er(q))^{n+1}$.
    We consider $p\in \er(q)$ and $a\in \sn$ which are associated. 
    % Recall that equation \ref{eq:fgradient} implies $f_a(q)$ satisfies
    % \[
    % \nabla_q f_a(q) = \frac{1}{c(q,a)}w(q,p), \quad c(q,a)\neq 0
    % \]
    Let 
    \[
        K(q) = \{(w_j)^n_{j=0} \mid w_j\in L^K_qM, \rho(q,w_j) > 1, \gamma_{q,w_j}(1)\in K\}
    \]
    and define on $K(q)$ the map
    \begin{align*}
        H:K(q)&\to K^{n+1}\\
        (w_j)^n_{j=0}&\mapsto (\gamma_{q,w_j}(1))^n_{j=0}.
    \end{align*}
    We will denote $p_j = \gamma_{q,w_j}(1) = \exp_q(w_j)$. Then by definition $p_j\in \er(q)$ and $w_j = \Omega(q,p_j)$. As $\rho$ is lower semi-continuous, we see that $K(q)\subset (L_q^KM)^n$ is open by an analogous argument to the one in the proof of \ref{prop:submanifolds}. As the exponential map is continuous, $H$ is also continuous. Furthermore as $\Omega:\mc{Z}\to L^+V$ is continuous and injective, we can construct a continuous inverse to $H$, making $H:K(q) \to H(K(q)) = (\er(q))^{n+1}$ a homeomorphism.
    We will denote $Y(q)\coloneqq(\varepsilon_U^{reg}(q))^{n+1}$.
    Note that for all $\widehat{q}\in \widehat{V}$, the data \ref{rmk:data} determine $\er(q)$ and thus also the set $Y(q)\subset K^n$, where $q=\F^{-1}(\widehat{q})$.
    
    Let us now consider the set 
    \[
        K_0(q) = \{(w_j)_{j=1}^n\in K(q) \mid w_1,\dots,w_n \text{ are linearly independent}\}.
    \]
    As linear independence is an open and non-degenerate property $K_0(q)$ is open and dense in $K(q)$. Since $H$ is a homeomorphism, $Y_0(q) = H(K_0(q))$ is open and dense in $Y(q)$ as well.
    
    We can now prove the final part of the proposition: Recall that given $C^0$-observation coordinates around $\widehat{q}$, we want to determine if they are also $C^\infty$-observation coordinates $\widehat{q}$.
    To that end, let $(W_\aa,\mathbf{s}_\aa)$ be $C^0$-observation coordinates around $\widehat{q}\in W_\aa$ with $q=\F^{-1}(\widehat{q})$. By definition we have $p_j\in \er(q)$ where $p_j=\mc{E}_{a_j}(q)$ are associated with $a_j$ and hence $(p_j)_{j=0}^n\subset Y(q)$. In the case where $(p_j)^n_{j=0}\in Y_0(q)$, by proposition \ref{prop:observationtimecoordinates}, $q$ has a neighborhood $V_1\subset M$ on which the function $\mathbf{f}_{\overrightarrow{a}}:V_1\to\R^n$ gives smooth local coordinates. Thus, after possibly restricting $W_{\overrightarrow{a}}$, $(W_{\overrightarrow{a}},\mathbf{s}_{\overrightarrow{a}})$ are $C^\infty$-observation coordinates around $\widehat{q}$. 
    We then let $(W_{\overrightarrow{b}},\mathbf{s}_{\overrightarrow{b}}), \overrightarrow{b}\in (\sn)^{n+1}$ be different $C^0$-observation coordinates around $\widehat{q}$ and let $(\widetilde{p}_j)^n_{j=0}\in Y(q)$ be such that $\widetilde{p}_j$ is associated to $b_j$.
    Since all smooth coordinates must be compatible, then $(\widetilde{p_j})^n_{j=0}\in Y_0(q)$ if and only if 
    \begin{multline}\label{eq:compatibility}
        \text{The function $\mathbf{s}_{\overrightarrow{b}}\circ \mathbf{s}_{\aa}^{-1}$ is smooth at $\mathbf{s}_{\overrightarrow{a}}(\widehat{q})$ and the Jacobian determinant}\\
        \det(D(\mathbf{s}_{\overrightarrow{b}}\circ \mathbf{s}_{\overrightarrow{a}}^{-1}))\text{ at } \mathbf{s}_{\overrightarrow{a}}(\widehat{q})\text{ is non-zero.}
    \end{multline}
    Here the \enquote{only if}-direction follows from the fact that the nondegeneracy of the Jacobian ensures that the linear independence of the spanning vectors is preserved.
    
    For some $\overrightarrow{p}=(p_j)^n_{j=0}\in Y(q)$ with $\aa$ associated we define $\mc{X}_{\overrightarrow{p}}\subset Y(q)$ to be the set of $(\widetilde{p}_j)^n_{j=0}\in Y(q)$, such that for the associated $\overrightarrow{b}$ there exists $W_{\overrightarrow{b}}$ such that $(W_{\overrightarrow{b}},\mathbf{s}_{\overrightarrow{b}})$ are $C^0$-coordinates around $\widehat{q}$ and condition \ref{eq:compatibility} is satisfied.
    
    If $\overrightarrow{p}\in Y_0(q)$ we see that $Y_0(q) \subset \mc{X}_{\overrightarrow{p}}$. On the other hand $\overrightarrow{p}\notin Y_0(q)$ we have $Y_0(q) \cap \mc{X}_{\overrightarrow{p}} =\emptyset$. Since the set $Y_0(q)$ is open and dense in $Y(q)$, we see that $\overrightarrow{p}\in Y_0(q)$ if and only if the interior of $\mc{X}_{\overrightarrow{p}}$ is dense subset of $Y(q)$. Since the data \ref{rmk:data} is sufficient to determine $Y(q)$ and $\mc{X}_{\overrightarrow{p}}$, we can determine whether $\overrightarrow{p}\in Y_0(q)$ or not. And since, by proposition \ref{prop:observationtimecoordinates}, the $C^0$-observation coordinates $(W_\aa,\mathbf{s}_\aa)$ around $\widehat{q}=\F(q)$ are $C^\infty$-observation coordinates if and only if $\overrightarrow{p}\in Y_0(q)$, where $\overrightarrow{p}$ are associated to $\aa$ wrt. $q$, we can determine all $C^0$-observation coordinates around $\widehat{q}$ which are also $C^\infty$-observation coordinates.
\end{proof}

\subsubsection{Construction of the conformal type of the metric}
We will denote by $\widehat{g}=\F_*g$ the metric on $\widehat{V}=\F$ that makes $\F:V\to \widehat{V}$ an isometry. Next we will show that the set $\F(V)$, the paths $\mu_a$ and the conformal class of the metric on $U$ determine the conformal class of $\widehat{g}$ on $\widehat{V}$.

\begin{lemma}\label{lem:constructmetric}
The data given in \ref{rmk:data} allows us to determine a metric $G$ on $\widehat{V}=\F(V)$ that is conformal to $\widehat{g}$ and a time orientation on $\widehat{V}$ that makes $\F:V\to \widehat{V}$ a causality preserving map.
\end{lemma}
\begin{proof}
    Let $(W_\aa,\mathbf{s}_\aa)$ be $C^\infty$-observation coordinates on $\widehat{V}$ and $\widehat{q}\in W_\aa$.
    We begin by constructing a time orientation on $\widehat{V}$:
    Let $a_1, a_2\in \aa$ and $p_1,p_2\in U$ be associated wrt. the point $q=\F^{-1}(\widehat{q})$, i.e. $p_i=\mc{E}_{a_i}(q)$.
    Because $\mathbf{f}_\aa = \mathbf{s}_\aa \circ \F$ are smooth coordinates we have that the vectors $w(q,p_1)$ and $w(q,p_2)$ pointing from $q$ to $p_i$ must be non-parallel. Therefore, by equation \ref{eq:fgradient} we see that the gradient vectors $\nabla f_{a_i}(q)$ are non-parallel, lightlike and past-pointing. 
    Thus the co-vectors $-ds_{a_1}\rvert_{\widehat{q}}$ and $-ds_{a_2}\rvert_{\widehat{q}}$ are non-parallel lightlike and future-pointing. This follows from the fact that $\F$ is an isometry and the co-vector $df_a$ is the image of  $\nabla f_a$ under the canonical isomophism. Moreover because the data allows us to fully determine $s_{a_1}$ and $s_{a_2}$ on $\widehat{V}$ (see previous proof) we can also determine $ds_{a_1}$ resp. $ds_{a_2}$.
    
    The co-vector field $X=(-ds_{a_1})+(-ds_{a_2})$ is timelike and future-pointing and forms a local time-orientation on $W_\aa$. Using bump functions and a partition of unity we can then obtain a time-orientation on the whole of $\widehat{V}$ since all orientations agree where they overlap.
    
    Now we turn our attention to the construction of a metric $G$ which is conformal to $\widehat{g}$: Let again $(W_\aa,\mathbf{s}_\aa)$ be $C^\infty$-observation coordinates on $\widehat{V}$ with $\widehat{q}_0\in W_\aa$ and $q_0\in V$ such that $\widehat{q}_0=\F(q_0)$. As in the previous proof, using the data given in $\ref{rmk:data}$ and the function $\widehat{q_0}=F_{q_0}$ we can determine $\ee(q_0), \er(q_0), \mc{D}_K(q_0)$ and $\mc{D}^{reg}_K(q_0)$ by \ref{prop:obsreconstr}.
    
    We then fix the point $\widehat{q_0}=\F(q_0)$ and the tuple $(p,v)\in \mc{D}^{reg}_K(q_0)$. Let $\widehat{t} >0$ be the largest number such that the geodesic $\gamma_{p,v}((-\widehat{t},0])\subset M$ is defined and has no cut point. For $q\in V$, we have that $q\in \gamma_{p,v}((-\widehat{t},0))$ if and only if $(p,v)\in \mc{D}^{reg}_K(q)$. Hence for a fixed $(p,v)\in \mc{D}^{reg}_K(q_0)$ the data allows us to whether some $\widehat{q}\in W_\aa$ has $q=\F^{-1}(\widehat{q}) \in \gamma_{p,v}((-\widehat{t},0))$ by checking if $(p,v)\in \mc{D}^{reg}(q)$. This allows us to determine
        \[
        \beta = \{ \widehat{q}\in W_\aa \mid \widehat{q}=\F(q), \mc{D}^{reg}_K(q) \ni (p,v)\} = \F(\gamma_{p,v}((-\widehat{t},0))) \cap W_\aa.
    \]
    Therefore, on $W_\aa\subset \widehat{V}$ we can find the image, under the map $\F$, of the light-like geodesic segment $\gamma_{p,v}((-\widehat{t},0))\cap \F^{-1}(W_\aa)$ that contains $q_0=\gamma_{p,v}(-t_1)$. Let $\alpha(s), s\in (-s_0,s_0)$ be a smooth path on $W_\aa$ such that $\partial_s\alpha(s)$ is never zero, $\alpha((-s_0,s_0))\subset \beta$ and $\alpha(0) = \widehat{q}_0$. Such a smooth path can, for example be obtained by endowing $\widehat{V}$ with some arbitrary Riemannian metric and parameterizing by arc-length. Then $\widehat{w}=\partial_s\alpha(s)\rvert_{s=0}\in T_{\widehat{q}_0}\widehat{V}$ has the form $\widehat{w}=c\F(\gamma'_{p,v}(-t_1))$ where $c\neq 0$. 
    
    Since we can do the above construction for all points $(p,v)\in \mc{D}^{reg}_U(q_0)$, we can determine in the tangent space $T_{\widehat{q}_0}\widehat{V}$ the set 
    \[
    \Gamma = \F_*(\{cw\in L_{q_0}M \mid \exp_{q_0}(w) \in \er(q_0), c\in \R\setminus \{0\}\}) 
    \]
    which is an open, non-empty subset of the light cone at $\widehat{q}_0$ wrt. the metric $\widehat{g}$. But now, since the light cone is determined by a quadratic equation in the tangent space, having an open set $\Gamma$ determines the whole light cone. By repeating this construction for all points $\widehat{q}\in \widehat{V}$, we can uniquely determine $L\widehat{V}$. Using proposition \ref{prop:metricfromnullcone} we can then determine the conformal class of the tensor $\widehat{g}=\F_*g$ in the manifold $\widehat{V}$.
    
    The above shows that the data \ref{rmk:data} determine the conformal class of the metric tensor $\widehat{g}$. And in particular we can construct a metric $G$ on $\widehat{V}$ that is conformal to $\widehat{g}$ and satisfies $G(X,X)=-1$.
\end{proof}

\begin{proof}
    We have now gone through all the steps necessary to reconstruct the conformal, differential and topological data of $V$, we will not tie this all together to give a detailed account of the actual reconstruction and prove theorem \ref{thm:reconstr}.
    
    Assume we are given data $\ref{rmk:data}$, that is, the smooth manifold $U$, the conformal class of $g\rvert_U$, the family of smooth and timelike future-pointing paths $\mu_a, a\in \ca$ and the collection of light observation sets $\mc{P}_U(V)=\{\mc{P}_U(q)\mid q\in V\}$. Note that for a given  $\mc{P}_U(q)$, we cannot, a priory, determine to which $q\in V$ it belongs.
    
    To then construct $\widehat{V}$ which is conformally diffeomorphic to $V$ we follow these steps:
    \begin{itemize}
        \item As $f_a(q) = \min \{s\in [-1,1] \mid \mu_a(s) \in \mc{P}_U(q)$ we can determine $\varepsilon_U(V) = \{\varepsilon_U(q) \mid q\in V\}$ from $\mc{P}_U(V)$.
        \item Proposition \ref{prop:recoverD} then allows us to determine $\mc{D}^{reg}_U(q)$, $\mc{D}_U(q)$ and $\varepsilon_U^{reg}(q)$ for a given $\varepsilon_U(q)\in \varepsilon_U(V)$. We can thus construct $\mc{D}^{reg}_U(V)$, $\mc{D}_U(V)$ and $\varepsilon_U^{reg}(V)$.
        \item We define the function 
        \begin{align*}
            \F:V&\to \F(V) = \widehat{V} \subset \R^{\ca}\\
            q &\mapsto \widehat{q} = F_q = (a\mapsto f_a(q)).
        \end{align*}
        For a given $\varepsilon_U(q)$ we can construct $\widehat{q}$ by $\widehat{q}(a) = f_a(q) = s$ such that $\mu_a(s) \in \varepsilon_U(q)$.
        This allows us to construct the following map
        \begin{align*}
            \widetilde{\F}:V&\to \widehat{V}\\
            \varepsilon_U(q) &\mapsto \widehat{q}.
        \end{align*}
        Note that by the same procedure we can define the map
        \begin{align*}
            \widetilde{f}_a:\varepsilon_U(V)&\to\R\\
            \varepsilon_U(q) &\mapsto f_a(q).
        \end{align*}
        
        Furthermore if we endow $\widehat{V}$ with the product topology and let $\pi_a:\widehat{V}\to \R$ with $\widehat{q}\mapsto \widehat{q}(a)$ then the following diagrams commute for all $a\in \ca$:
        \[\begin{tikzcd}
        V &&& {\widehat{V}} \\
        {\varepsilon_U(V)} & {\widehat{V}} & V & {\mathbb{R}} & {\varepsilon_U(V)}
        \arrow["{\mathcal{F}}", from=1-1, to=2-2]
        \arrow["{\widetilde{\mathcal{F}}}"', from=2-1, to=2-2]
        \arrow["{q\mapsto\varepsilon_U(q)}"', from=1-1, to=2-1]
        \arrow["{f_a}"', from=2-3, to=2-4]
        \arrow["{\widetilde{f}_a}", from=2-5, to=2-4]
        \arrow["{\mathcal{F}}", from=2-3, to=1-4]
        \arrow["{\widetilde{\mathcal{F}}}"', from=2-5, to=1-4]
        \arrow["{\pi_a}", from=1-4, to=2-4]
    \end{tikzcd}\]
    
    Note that in the above diagrams, the data \ref{rmk:data} only allows us to construct the sets $\varepsilon_U(V)$ and $\widehat{V}$ and the maps $\widetilde{F}$, $\widetilde{f}_a$ and $\pi_a$.
    
    We thus can construct the space $\widehat{V}$ and endow it with the product topology since both $\ca$ and $\R$ are metric spaces.
    
    \item We can then reconstruct the topological structure of $V$, since lemma \ref{lem:Fhomeo} shows that $\F:V\widehat{V}$ is a homeomorphism and we know the topology of $\widehat{V}$.
    
    \item For a given point $\widehat{q}\in \widehat{V}$ we can use proposition \ref{prop:findsmoothcoords} and the date \ref{rmk:data} to determine all $C^0$-observation coordinates around $\widehat{q}$. We can then determine for each of these coordinates if they are also $C^\infty$-observation coordinates, and find at least one such coordinates since existence is guaranteed. We then repeat that step for each $\widehat{q}\in \widehat{V}$ to find smooth coordinates on $\widehat{V}$. Since these coordinates make $\F$ a diffeomorphism this allows us to recover the differential structure of $V$.
    
    \item Finally we use lemma \ref{lem:constructmetric} to construct a metric $G$ and time-orientation $X$ on $\widehat{V}$ which is conformal to $\widehat{g}=\F_*g$ and makes $\F$ causal. 
    $\F:(V,g\rvert_U)\to (\widehat{V},G)$ is thus a causal diffeomorphism and we can reconstruct the causal and conformal structure of $V$.
    \end{itemize}
    
    Given the data \ref{rmk:data}, we were able to reconstruct the conformal, differential and topological structure of $V$ and since the data is invariant under conformal diffeomorphism, the first consequence of theorem \ref{thm:reconstr} follows.
    
    For the final part of theorem \ref{thm:reconstr}, by \ref{prop:recoverD}(1) we can identify all $q\in U\cap V$ using data \ref{rmk:data} which uniquely determines the map.
\end{proof}