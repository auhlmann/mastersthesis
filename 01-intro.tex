\chapter{Introduction}

\section{Main Results}
((Introduce Notation etc.))

\begin{theorem}[Interior Reconstruction]\label{thm:intreconstr}
    Let $(M_j,g_j), j=1,2$ be two open globally hyperbolic, time-oriented Lorentzian manifolds. For $p_j^-\ll p_j^+$ two points in $M_j$ we denote $K_j = J(p_j^-,p_j^+) \setminus I^-(p^+_j)$, the closed and compact backwards light cone from $p_j^+$ cut off at the intersection with the forwards light cone of $p_j^-$. We assume that there exists a conformal diffeomorphism $\Phi:K_1\to K_2$ and that none of the past null geodesics starting at $p_j^+$ have a cut point in $K_j$. 
    
    Now let $V_j\subset\operatorname{int} J(p_j^-,p_j^+)$ be open sets. We assume that no null geodesic starting in $V_j$ has a null conjugate point on $K_j$. 
    
    Then, if 
    \[
    \widetilde{\Phi}(\mc{P}_{K_1}(V_1)) = \mc{P}_{K_2}(V_2)
    \]
    there exists a conformal diffeomorphism $\Phi:V_1\to V_2$ that preserves causality.
\end{theorem}


\begin{theorem}[Boundary Reconstruction]\label{thm:bdreconstr}
    Let $(M_j,g_j), j=1,2$ be two open globally hyperbolic, time-oriented Lorentzian manifolds. For $p_j^-\ll p_j^+$ two points in $M_j$ we denote $K_j = J(p_j^-,p_j^+) \setminus I^-(p^+_j)$, the closed and compact backwards light cone from $p_j^+$ cut off at the intersection with the forwards light cone of $p_j^-$. We assume that there exists a conformal diffeomorphism $\Phi:K_1\to K_2$ and that none of the past null geodesics starting at $p_j^+$ have a cut point in $K_j$. 
    
    Now let $V_j\subset J(p_j^-,p_j^+) \setminus p_j^+$ be open sets. We assume that no null geodesic starting in $V_j$ has a null conjugate point on $K_j$. 
    
    Then, if 
    \[
    \widetilde{\Phi}(\mc{P}_{K_1}(V_1)) = \mc{P}_{K_2}(V_2)
    \]
    there exists a conformal diffeomorphism $\Phi:V_1\to V_2$ that preserves causality.
\end{theorem}