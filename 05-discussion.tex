\chapter*{Further directions}
Finally outline some interesting further directions for study: First of all, analogously corollary 1.3 in \cite{kurylev2017inverse} it would be interesting to study whether in the boundary reconstruction case where $V$ and $K$ might overlap, it is possbible to even construct the metric $g$ on $V$ itself (and not only up to a conformal factor).

In line with the active reconstruction results by \cite{wang2019inverse} and \cite{lassas2018inverse}, it would be interesting to investigate whether such a result could also be obtained in our case, i.e. if for some nonlinear wave equations on $(M,g)$ we could reconstruct $\interior{\jp}$ from knwoledge of the source-to-solution map, mapping sources on $K^-\coloneqq  (J^+(p^-)\setminus I^+(p^-)) \cap J^-(p^+)$ to observations on $K$.

Furthermore during our investiation of reconstruction on the Einstein universe the question arose whether a Lorentzian manifold $(\R^{1+n},g)$ with no conjucate point must necessarily be flat. As shown by \citet{guillarmou2019asymptotically}, this is the case for asymptotically flat Riemannian manifolds.