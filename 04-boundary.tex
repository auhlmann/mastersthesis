\chapter{Boundary Reconstruction}

\section{Setting}

In this section we will examine how we can extend our reconstruction result to the case where the observed set $V$ is no longer contained within the interior of $J(p^-,p^+)$ but is now allowed to extend up to the boundary. In other words we want to recover the conformal structure of $J(p^-,p^+)$ from light cone observations made on the future null boundary $K = J(p^-,p^+) \setminus I^-(p^+)$.

This setting is complicated by the fact that as $q\in J(p^-,p^+)$ approaches the boundary, the light observation $\mc{P}_K(q)$ get increasingly warped and is degenerate if $q$ is in the boundary. 


\begin{theorem}[Boundary Reconstruction]\label{thm:boundaryreconstr}
    Let $(M_j,g_j), j=1,2$ be two open globally hyperbolic, time-oriented Lorentzian manifolds. For $p_j^-\ll p_j^+$ two points in $M_j$ we denote $K_j = J(p_j^-,p_j^+) \setminus I^-(p^+_j)$, the closed and compact backwards light cone from $p_j^+$ cut off at the intersection with the forwards light cone of $p_j^-$. We assume that there exist a conformal diffeomorphism $\Phi:K_1\to K_2$ and that none of the past null geodesics starting at $p_j^+$ have a cut point in $K_j$. 

    Now let $V_j\subset J(p_j^-,p_j^+)$. We assume that no null geodesic starting in $V_j$ has a conjugate point on $K_j$. 

    Then, if 
    \[
    \widetilde{\Phi}(\mc{P}_{K_1}(V_1)) = \mc{P}_{K_2}(V_2)
    \]
    there exists a homeomorphism $\Phi:V_1\to V_2$.((conformal diffeomorphism $\Phi:V_1\to V_2$ that preserves causality))
\end{theorem}

\begin{remark}
    ((Move this to intro))
    We will again prove the previous theorem by working on just one globally hyperbolic Lorentzian manifold $(M,g)$ with $p^\pm,V,K$ as above. We show that given the following data we can reconstruct the topology of $V$:

    \begin{enumerate}[label={\textnormal{(\arabic*)}}]
        \item The smooth manifold (with edge) $K$,
        \item the conformal class of $g\rvert_K$,
        \item the set $\P_K(V)$.
    \end{enumerate}
\end{remark}

\section{Preliminaries}
To extend the reconstruction up to the edge of $\jp$ we need to introduce some new concepts.

\begin{definition}[Unique minimum domain]\label{def:uniquemindomain}
    We define the \emph{unique minimum domain} $D\subset \jp$ to be 
    \begin{equation}
        D:=\{q\in \jp \mid F_q\text{ has a unique minimum}\}.
    \end{equation}

    We will often describe this minimum with 
    \[
        (a_q,t_q)=(\argmin_{a\in \sn} F_q, \min_{a\in \sn} F_q).
    \]
\end{definition}

\begin{remark}
    To account for the fact that we now also need to reconstruct points $q\in K$ which lie on be boundary we will introduce a second type of coordinates on $D$ related to the unique minimum.
    
    Let $V\subset \jp\setminus \{p^+\}$. As we saw in the previous chapter we can reconstruct $V\cap \interior{\jp}$. In this chapter we will show that we can also reconstruct $V\cap D$. Furthermore we will show that $D$ is open and $D\cup \interior{\jp} = \jp  \setminus \{p^+\}$. We can then combine this to reconstruct all of $V$.
\end{remark}


\begin{definition}[Constant observation time domain] For some $t_0\in (0,1)$ we define the \emph{constant observation time domain} as 
\begin{equation}
    T_{t_0} = \{p\in K \mid p=\mu_a(t_0), a\in \sn\} = \Theta(\sn\times \{t_0\}) \subset K.
\end{equation}
Note that because $\Theta$ is a diffeomorphism on $\sn \times (0,1)$ and by definition, $T_{t_0}$ is a $n-1$ dimensional spacelike submanifold of $K$. 
Thus for every $a\in \sn$ we can use lemma \ref{lem:dirreconstr} to find the unique future-pointing outward null ray $\R_+\nu_a$ such that $T_{\Theta(a,t_0)}T_{t_0} = \nu_a^\perp \cap T_{\Theta(a,t_0)}K$.
\end{definition}

\begin{lemma}
    Let $q\in V\cap D$ with minimum $(a_q,t_q)$ and $p_q:=\Theta(a_q,t_q)$. Then we have $p_q\in \er(q)$. And if we let $w_q\in L^K_qM$ be the unique null vector such that $\gamma_{q,w_q}(1)=p$ we have $\gamma'_{q,w_q}(1)=\nu_{a_q}$.
\end{lemma}
\begin{proof}
    Note that we have $t_q = f_{a_q}(q)$ and thus 
    \[
        p_q = \Theta(a_q,t_q)=\mu_{a_q}(t_q) = \mu_{a_q}(f_{a_q}(q)) = \mc{E}_{a_q}(q)\in \pq,
    \] proving that there exists a $w_q\in L^K_qM$ such that $p_q = \gamma_{q,w_q}(1)$. 

    Now we need to show that indeed $p_q\in \er(q)$.
    We recall that by prop \ref{prop:unionmanif} there exists an open neighborhood $p_q\in U\subset M$ such that $\mc{P}_K(q)\cap U$ is the union of $N$ pairwise transversal, spacelike, dimension $n-1$ submanifolds $\mc{V}_i$. Because $t_q$ is the minimum of $F_q$ we must have $T_{p_q}{\mc{V}_i} = T_{p_q}{T_{t_q}}$ for all $i=1,\dots N$. But because the manifolds must be pairwise transversal, we must have $N=1$, implying that $p_q$ is a regular point. Together with $p_q\in \ee(q)$ this yields $p_q \in \er(q)$.

    Finally $\gamma'_{q,w_q}(1)=\nu_{a_q}$ follows from the fact that $T_{p_q}{\mc{V}_1} = T_{p_q}{T_{t_q}}$.
\end{proof}

