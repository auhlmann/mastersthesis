\chapter{Applications}
In this chapter we will examine some useful applications of the results proven in the previous chapters. We first show that the suitability conditions required in theorems \ref{thm:intreconstr} and \ref{thm:bdreconstr} are somewhat generic as they are preserved by small perturbations. Then we will use a conformal embedding of Minkowski space into the Einstein universe to show that theorem \ref{thm:intreconstr} also allows us to reconstruct the complete spacetime from observations at future null infinity.

\section{Stability Results}
The following lemma guarantees that theorem \ref{thm:intreconstr} still applies even if we deviate the metric slightly:
\begin{lemma}
    Let $(M,g)$ be a globally hyperbolic manifold with $p^-,p^+\in M$ \emph{suitable}. Furthermore let $V\in \interior{\jp}$ such for any $q\in \cl{V}$, no null geodesic starting at $q$ has a conjugate point in $K$.

    If we vary the metric $g$ slightly to $\widetilde{g}:=g+h$ such that 
    \[
        \lvert h_{ij}\rvert_q \rvert <\varepsilon, \quad \lvert h_{ij,\alpha}\rvert_q \rvert <\varepsilon, \quad \lvert h_{ij,\alpha,\beta}\rvert_q \rvert <\varepsilon \quad \text{for }i,j,\alpha,\beta\in \{1,\dots, 1+n\}
    \]
    and $\widetilde{g}$ is smooth and has $h_q=0$ for all $q\in M\setminus V$. Then if $\varepsilon>0$ is small enough, $(M,\widetilde{g})$ is globally hyperbolic and $p^-,p^+,V$ are still \emph{suitable}
\end{lemma}
\begin{proof}
    To distinguish between objects defined in terms of $g$ or $\widetilde{g}$ we will add a prescript of the respective metric, for example $\psg{\exp}_q$ is the exponential map defined with respect to $g$ and $\psgt{\exp}_q$ with respect to $\widetilde{g}$. We begin by mentioning that for $\varepsilon>0$, small enough, $(M,\widetilde{g})$ still globally hyperbolic. This follows from theorem 12 in \cite{gerochdomain}.
    
    Next we want to show that we also have $p^- \psgt{\ll} p^+$, i.e. there exists a timelike (wrt. $\widetilde{g}$) path from 
    $p^-$ to $p^+$. Because $p^- \psg{\ll} p^+$, there exists a path $\sigma$ with $\sigma(0)=p^-$ and $\sigma(1)=p^+$ which is timelike wrt. $g$. Because $[0,1]$ is compact we can find a $\varepsilon_2>0$ such that $\sigma$ is still timelike wrt. $\widetilde{g}$, and thus $p^-\ll g^+$.

    Because $p^-\ll p^+$ and $(M,\widetilde{g})$ globally hyperbolic, the map $a\in \sn \mapsto \psgt{T}_a\in (0,\infty)$ as in lemma \ref{lem:Tvdef} is well-defined and continuous making $\psgt{\mc{S}} := \{(a,t)\in \sn\times [0,\infty) \mid t\in [0,\psgt{T}_a]\}$ a compact set. 

    Now because the geodesic equation on $(M,\widetilde{g})$ is a second order ODE with coefficients $\widetilde{g}_{ij}$ and $\widetilde{g}_{ij,\alpha}$ and $h=\widetilde{g}-g$ has compact support, $\psgt{\exp}:TM\to M$ depends smoothly on $\widetilde{g}_{ij}$ and $\widetilde{g}_{ij,\alpha}$ while $d\psgt{\exp}:T(TM)\to TM$ depends smoothly on $\widetilde{g}_{ij}$, $\widetilde{g}_{ij,\alpha}$ and $\widetilde{g}_{ij,\alpha,\beta}$ for some $\varepsilon_3>0$ small enough. This also implies that $\psgt{\rho(q,w)}$ as well as $\psgt{T}_a$ depend smoothly on $\widetilde{g}$ and its first and second derivatives. Because we have $\psg{\rho(p^+,a)}>\psg{T}_a$ for all $a\in \sn$ there exists a $\varepsilon_4>0$ such that $\psgt{\rho(p^+,a)}>\psgt{T}_a$ for all $a\in \sn$ and we have proved that $p^-,p^+$ are suitable.

    Note that because $\widetilde{g}=g$ outside of $V$, we still have $V\subset \interior{\psgt{\jp}}$. We can then see that $p^-,p^+,V$ are still suitable with respect to $\widetilde{g}$ and some $\varepsilon_5>0$ after noting that $\psgt{L}^K\cl{V}$ is compact and $\psg{\exp}$ has no conjugate points in $\psg{L}^K\cl{V}$ by assumption.
\end{proof}

We can extend the previous result slightly to show that if also the past boundary of $\jp$ has no cut points, then $\jp$ and $K$ are conserved.
\begin{corollary}\label{cor:suitablestable}
    Let $(M,g)$ be a globally hyperbolic manifold with $p^-,p^+\in M$ \emph{suitable} and such that no geodesic starting at $p^-$ has a cut point in $\mc{L}^+_{p^-}\cap J^-(p^+)$. Furthermore let $V\in \interior{\jp}$ such for any $q\in \cl{V}$, no null geodesic starting at $q$ has a conjugate point in $K$.
    
    If we vary the metric $g$ slightly to $\widetilde{g}:=g+h$ such that 
    \[
        \lvert h_{ij}\rvert_q \rvert <\varepsilon, \quad \lvert h_{ij,\alpha}\rvert_q \rvert <\varepsilon, \quad \lvert h_{ij,\alpha,\beta}\rvert_q \rvert <\varepsilon \quad \text{for } i,j,\alpha,\beta\in \{1,\dots, 1+n\}
    \] $q\in \interior{\jp}$
    and $\widetilde{g}$ is smooth and has $h=0$ for all $q\in M\setminus \interior{\jp}$. Then if $\varepsilon>0$ is small enough, $(M,\widetilde{g})$ is globally hyperbolic and $p^-,p^+,V$ are still \emph{suitable}. Furthermore we have $\psg{\jp}=\psgt{\jp}$ and $\psg{K}=\psgt{K}$.
\end{corollary}
\begin{proof}
    The proof follows from the observation that the fact that $p^+$ and $p^-$ have no cut points in $\mc{L}^\mp_{p^\pm} \cap J^\pm(p^\mp)$ implies $\psg{\exp_{p^\pm}}= \psgt{\exp_{p^\pm}}$, together with an analogous argument to the previous lemma.
\end{proof}

\begin{example}
    Because Minkowski space $(\R^{1+n}, g_M:=-dt^2 + \sum dx^2)$ has no cut points. We can pick any $p^-,p^+,V$ \emph{suitable} such that $V\in \interior{\jp}$ using the previous lemma we can see that for small deviatations from $g_M$ with support on $V$, theorem \ref{thm:intreconstr} still applies and we can reconstruct $V$ from the light cone observations on $K$.
\end{example}



However this example is somewhat limited in scope because such a deviation cannot be physical. To get a physical example we will use the reconstruction result on the Einstein universe:
% \begin{proposition}
% Let $K\subset\R^n_1$ be a compact subset of the $1+n$-dimensional minkovsky space with metric $g=-dt^2+\sum_{i=1}^n dx_i^2$. And let $\widetilde{g}$ be a slightly disturbed metric $\widetilde{g}=g+\varepsilon h$ where $\varepsilon>0$ and $h$ is another metric.

% Then we can choose $\varepsilon>0$ small enough such that under the disturbed metric $\widetilde{g}$ no causal geodesic starting in $K$ has a conjugate point in $K$.
% \end{proposition}
% \begin{proof}
% We begin by defining the set $H = \{(p,v)\in TK \mid \exp_p(v) \in K\}$. Note that for $\varepsilon>0$ small enough this set is compact as well. This is because in the minkovsky case, every geodesic starting in $K$ leaves $K$ in a finite time which depends continuously on the starting point and initial direction. This property still holds for $\widetilde{g}$ if $\varepsilon$ is small enough and thus $H$ is compact.

% Recall that ((REF GEOD)) for a geodesic $\gamma_{p,v}$ starting at $p$ with initial velocity $v$, $q=\gamma_{p,v}(b)$ is a conjugate point of $p=\gamma_{p,v}(0)$ if and only if the differential of the exponential map $d\exp_p$ is singular at $bv$. 
% We we then define 
% \[
%     \varepsilon_{p,v} = \frac{1}{2}\sup\{\varepsilon'>0\mid d\exp\}
% \]((Bla bla bla, do it with open cover instead))
% By compactness of $H$ we can achieve that $\exp$ is never singular on it which means that no geodesic starting in $K$ has a conjugate point in $K$. 
% \end{proof}
% ((Is global hyperbolicity needed?))
% Note that this proof can be directly generalized to show that if $K$ is a compact subset of a globally hyperbolic manifold $(M,g)$ and every causal geodesic starting in $K$ has no conjugate point in $K$, then also for a slightly perturbed $(M,g+\varepsilon h)$, causal geodesics starting in $K$ have no conjugate points in $K$.
% ((Expand conj point to cut points?))

\section{Einstein Universe}

\begin{definition}[Einstein Universe]
Let $(\R,-dt^2)$ be the real line with negative definite metric $-dt^2$ and $(S^n,h)$ the $n$-sphere with the canonical Riemannian metric. The $1+n$ dimensional \emph{Einstein universe} is then defined as the product $(\R\times S^n, -dt^2 \oplus h)$. Note that by construction is a Lorentzian globally hyperbolic manifold.
\end{definition}

To better describe the Einstein universe the following remark is very useful
\begin{remark}\label{rmk:alphadef}
We can parameterize $S^n$ by an angle $\alpha\in [0,\pi]$ and a point $\omega\in S^{n-1}$ via the map 

\begin{align*}
    S:[0,\pi]\times S^{n-1}&\to S^n \\
    (\alpha,\omega) &\mapsto (\cos \alpha, \sin \alpha \omega)
\end{align*}

If for a $X\in S^n$ we write $X=(X_0,\overrightarrow{X}), X_0\in \R, \overrightarrow{X}\in \R^n$. We can invert $S$ by 
\[
    \alpha = \arccos X_0, \quad \omega=\frac{\overrightarrow{X}}{\lVert\overrightarrow{X}\rVert}.
\]

$S$ is surjective and smooth but we have 
\[(1,0,\dots, 0) = S(0,\omega) \quad \text{and} \quad (-1,0,\dots,0)=S(\pi,\omega) \quad \text{for all }\omega\in \sn,\]
which means $S$ fails to be injective if $\alpha = \{0,\pi\}$. Nontheless for every $X\in S^n$, $X\mapsto \alpha$ is well defined and smooth.
\end{remark}

\subsection{Conformal Embedding}
We will now describe how Minkowsky space can be conformally embedded into the Einstein universe and how we can find null the corresponding future and past infinities.
\begin{definition}
    We can first define $M^M$, the image of the conformal embedding and thus a conformal copy of Minkowski space within the Einstein universe:
    \begin{align*}
        M^M:=&\{(T,X)\in \R\times S^n \mid T\in(-\pi,\pi), \alpha<\pi-\lvert T\rvert\}
    \end{align*}
    Next we can define the future and past null infinities of $M^M$ to be (almost) its future and past boundaries:
    \begin{align*}
        \mc{J}^+:=&\{(T,X)\in \R\times S^n \mid T\in(0,\pi), \alpha = \pi - T\}\\
        \mc{J}^-:=&\{(T,X)\in \R\times S^n \mid T\in(\pi,0), \alpha = \pi + T\}
    \end{align*}
    As mentioned $\mc{J}^\pm$ are only \emph{almost} the full boundary of $M^M$; in fact $\{T=\pm\pi,\alpha=0\}$ and $\{T=0,\alpha=\pi\}$ are missing. But as remarked in \ref{rmk:alphadef}, at these points the mapping $X\mapsto \alpha$ is degenerate and 
    \begin{align*}
        i^\pm :=& \{T=\pm\pi,\alpha=0\} \\
        i^S := & \{T=0,\alpha=\pi\}
    \end{align*}
    are single points which correspond exactly to timelike future, timelike past and spacelike infinities.
\end{definition}

If we set $p^\pm = i^\pm$ we can see that $\mc{J}^+ = K \setminus (\{p^+\} \cup R)$, and that they are \emph{almost} suitable. A slight issue is that past null geodesics starting at $i^+=(\pi, 0)$ have a conjugate point at $i^s=(0,\pi)\in R$. We even have $R = \{i^S\}$. Meaning the past light cone starting at $i^+$ fully collapses into $i^s$. To fix this we can again argue that by lemma \ref{lem:Kcharact}(3) we can disregard this boundary behaviour and because no geodesic starting at $i^+$ has a conjugate point in $\mc{J}^+$ we can continue as if $i^-$ and $i^+$ were suitable.

We can now construct our conformal embedding:
\begin{proposition}
Let $(M^E,g_E)=(\R\times S^n,-dt^2 \oplus h)$ be the $1+n$ dimensional Einstein universe and $(\R^{1+n},h=dt^2-dx_ndx^n)$ the $1+n$ dimensional Minkovski space. Then the map 
\begin{align}
    \Psi:M^M &\to \R^{1+n}\\
    (T,X) &\mapsto\frac{1}{\cos T + X_0}(\sin T,\overrightarrow{X})
\end{align}
is a conformal diffeomorphism from $M^M$ to the whole Minkovski space $\R^{1+n}$ with conformal factor $\cos(T)+X_0$. Here $X_0$ denotes the first coordinate of $X$ under the canonical embedding of $S^n$ into $\R^{n+1}$.
\end{proposition}
\begin{proof}
    This can be verified by calculation. An extensive treatment can be found in \cite{hormander}(A.4).
\end{proof}

\begin{figure}\label{fig:EU}
    \centering
    \begin{tikzpicture}[auto,scale=1]
    \def\vtick#1#2{
        \draw ($#1+(-0.1,0)$) node[left]{#2} -- ($#1+(0.1,0)$);
    }
    \begin{scope}[on above layer]
        \def\da{0.3}
        \def\db{0.45}
        \path (-2,3) coordinate (A) (-2,-3) coordinate (B);
        \path (2,3) coordinate (C) (2,-3) coordinate (D);

        \draw ($(A)+(0,\da)$)--coordinate (M1) ($(B)+(0,-\da)$);
        \draw ($(C)+(0,\da)$)--coordinate (M2) ($(D)+(0,-\da)$);
        \draw[dotted,thick] ($(A)+(0,\da)$)--($(A)+(0,\da+\db)$) ($(B)+(0,-\da)$)--($(B)+(0,-\da-\db)$);
        \draw[dotted,thick] ($(C)+(0,\da)$)--($(C)+(0,\da+\db)$) ($(D)+(0,-\da)$)--($(D)+(0,-\da-\db)$);
    \end{scope}

    \begin{scope}
        \clip (-2,0) rectangle (2,-1);
        \draw (0,0) ellipse(2 and 0.2);
    \end{scope}
    \begin{scope}
        \clip (-2,0) rectangle (2,1);
        \draw[dashed] (0,0) ellipse(2 and 0.2);
    \end{scope}

    \vtick{(A)}{$\pi$}
    \vtick{(M1)}{$0$}
    \vtick{(B)}{$-\pi$}

    \begin{scope}[red, thick]
        \draw[on above layer] (-2,1.5) to[bend left=10] (0,3) node[point]{} node[above]{$i^+$} to[bend left=10] node[above right]{$\mathcal{J}^+$} (2,1.5);
        \draw[dashed] (-2,1.5) to[bend right=10] (0,0.2) to[bend right=10] (2,1.5);
    \end{scope}

    \begin{scope}[blue,thick]
        \draw[on above layer] (-2,-1.5) to[bend right=10] (0,-3) node[point]{} node[below]{$i^-$} to[bend right=10] node[below right]{$\mathcal{J}^-$} (2,-1.5);
        \draw[dashed] (-2,-1.5) to[bend left=10] (0,0.2) to[bend left=10] (2,-1.5);
    \end{scope}
    
    \path (0,1) node{$M^M$};

    \fill[opacity=0.1] (-2,1.5) to[bend left=10] (0,3) node[point]{} to[bend left=10] (2,1.5) -- (2,-1.5) to[bend left=10] (0,-3) to[bend left=10] (-2,-1.5) -- cycle;

    \fill[opacity=0.1] (-2,1.5) to[bend right=10] (0,0.2) to[bend right=10] (2,1.5) -- (2,-1.5) to[bend right=10] (0,0.2) to[bend right=10] (-2,-1.5) -- cycle;

    \coordinate (O) at (3,2);

    \draw (O) ellipse (0.5 and 0.2);
    \path (3.5,2) node[right]{$S^n$};

    \draw[white, line width=3pt] (O) -- ++(0,0.5);
    \draw[->] (O) -- ++(0,0.5) node[above]{$\mathbb{R}$};

\end{tikzpicture}
    \caption{
        Illustration of the Einstein universe together with the conformal copy of Minkovski space $M^M$ and the corresponding infinities $\mc{J}^\pm$ and $i^\pm$. We can see that $\mc{J^\pm}$ are codimension 1 null submanifolds of the Einstein universe and $i^\pm$ are single points.
    }
\end{figure}
Note that this allows us to identify $J^\pm$ with the future resp. past null infinities of Minkowski space, whereas $i^\pm$ are the future resp. past timelike infinities.
We can thus understand a reconstruction of a set $V\subset \interior{J(i^-,i^+)}$ from light observations on $\mc{J}^+$ as a reconstruction of a subset of Minkowsky space from observations at future null infinity. Notably this remains possible even if we vary $g$ slightly:
\begin{example}Let now $V:= \{(T,X)\in M^M \mid T>0\}$ and $p^\pm = i^\pm$ Then we we have $V\in \interior{\jp}$ and no null geodesic starting in $\cl{V}$ has cut point on $K=\mc{L}^-_{p^+}\cap J^+(p^-) \approx \mc{J}^+$. This is because for any $\omega \in S^n$ and $\eta\in T_\omega S^n$, $-\omega$ is a conjugate point of full dimension along $\gamma_{\omega,\eta}$ and in fact the first cut point. Furthermore by the same argument we can see that for all $v\in CL^-_{i^+}M$, $\gamma_{i^+,v}$ has a conjucate point of maximal dimension exactly where it intersects with $J^+(p^-)\setminus I^+(p^-)$, i.e. on the boundary $R$. Because of this $R$ is now a single point and not a dimension $n-1$ submanifold anymore. By lemma \ref{lem:Kcharact}(3) assures that this is no problem as all reconstruction only requires $K$ to be regular on $K\cap I^+(p^-)$.

We can thus assume that $p^-,p^+$ and $V$ are suitable and can apply corollary \ref{cor:suitablestable} to show that even if we vary $g$ slightly to $\widetilde{g}$ on $\interior{\jp}$ we can still carry out the reconstruction. 

If we then use $\Psi$ to push the whole situation into $\R^{1+n}$ we can see that $\Psi_*g$ is a slight variation of the Minkowski metric which does not necessarily have compact support. If we view $\mc{}^+$ as future null infinity we can see that we are able to reconstruct $\Psi(V) = \{(t,x)\in \R^{1+n} \mid t>0\}$ from the light observations at null infinity.

Using theorem 3.5 in \cite{friedrichspacetime} we can see that there even are families of physical spacetimes, i.e. satisfying the Einstein field equations, which can be reconstructed in this way.
\end{example}
