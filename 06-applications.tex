\section{Applications}

\begin{definition}[Einstein Universe]
Let $(\R,-dt^2)$ be the real line with negatively definite metric $-dt^2$ and $(S^n,h)$ the n-sphere with the canonical Riemannian metric. The $1+n$ dimensional \emph{Einstein universe} is then defined as the product $(\R\times S^n, -ds^2 \oplus h)$
\end{definition}

\begin{remark}
We can parameterize $S^n$ by an angle $\alpha\in (0,\pi)$ and a point $\omega\in S^{n-1}$ via the map 

\begin{align*}
    S:(0,\pi)\times S^{n-1}&\to S^n \\
    (\alpha,\omega) &\mapsto (\cos \alpha, \sin \alpha \omega)
\end{align*}

If for a $X\in S^n$ we write $X=(X_0,\overrightarrow{X}), X_0\in \R, \overrightarrow{X}\in \R^n$. We can invert $S$ by 
\[
    \alpha = \arccos X_0, \quad \omega=\frac{\overrightarrow{X}}{\lVert\overrightarrow{X}\rVert}.
\]

Note that $S$ has the irregular points $(\pm 1,0\dots,0)$
\end{remark}

We can now construct our conformal embedding:

\begin{proposition}
Let $(\R\times S^n,g)$ be the $1+n$ dimensional Einstein universe and $(\R^{1+n},h=dt^2-dx_ndx^n)$ the $1+n$ dimensional Minkovski space. Then the map 
\begin{align}
    \Psi:\R\times S^n &\to \R^{1+n}\\
    (T,X) &\mapsto\frac{1}{\cos T + X_0}(\sin T,\overrightarrow{X})
\end{align}
is a conformal diffeomorphism from a suitable subset $U\subset \R\times S^n$ to the whole Minkovski space.
\end{proposition}
\begin{proof}
\end{proof}

\section{Stability Results}
((Overhaul, include globally hyperbolic stability))
We aim to show the reconstruction result in a simplified case and first establish that small deviations from the minkovsky metric on a compact set introduce no conjugate points:
\begin{proposition}
Let $K\subset\R^n_1$ be a compact subset of the $1+n$-dimensional minkovsky space with metric $g=-dt^2+\sum_{i=1}^n dx_i^2$. And let $\widetilde{g}$ be a slightly disturbed metric $\widetilde{g}=g+\varepsilon h$ where $\varepsilon>0$ and $h$ is another metric.

Then we can choose $\varepsilon>0$ small enough such that under the disturbed metric $\widetilde{g}$ no causal geodesic starting in $K$ has a conjugate point in $K$.
\end{proposition}
\begin{proof}
We begin by defining the set $H = \{(p,v)\in TK \mid \exp_p(v) \in K\}$. Note that for $\varepsilon>0$ small enough this set is compact as well. This is because in the minkovsky case, every geodesic starting in $K$ leaves $K$ in a finite time which depends continuously on the starting point and initial direction. This property still holds for $\widetilde{g}$ if $\varepsilon$ is small enough and thus $H$ is compact.

Recall that ((REF GEOD)) for a geodesic $\gamma_{p,v}$ starting at $p$ with initial velocity $v$, $q=\gamma_{p,v}(b)$ is a conjugate point of $p=\gamma_{p,v}(0)$ if and only if the differential of the exponential map $d\exp_p$ is singular at $bv$. 
We we then define 
\[
    \varepsilon_{p,v} = \frac{1}{2}\sup\{\varepsilon'>0\mid d\exp\}
\]((Bla bla bla, do it with open cover instead))
By compactness of $H$ we can achieve that $\exp$ is never singular on it which means that no geodesic starting in $K$ has a conjugate point in $K$. 
\end{proof}
((Is global hyperbolicity needed?))
Note that this proof can be directly generalized to show that if $K$ is a compact subset of a globally hyperbolic manifold $(M,g)$ and every causal geodesic starting in $K$ has no conjugate point in $K$, then also for a slightly perturbed $(M,g+\varepsilon h)$, causal geodesics starting in $K$ have no conjugate points in $K$.
((Expand conj point to cut points?))
