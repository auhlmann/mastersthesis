\chapter{Geometric Preliminaries}

\section{Null Conjugate Points}
((Leave this here?))
\begin{definition}[Null Conjugate Point]
    Let $\gamma_{q,w}:[0,b]\to M$ be a null geodesic. We then call $p=\gamma_{q,w}(b)$ a \emph{null conjugate point} if there exists a nontrivial variation $\x:[0,b]\times(-\varepsilon,\varepsilon) \to M$ of $\gamma_{q,w}$ through null geodesics such that $\x_v(b,0)=0$.
\end{definition}

We have the following useful characterization:
\begin{proposition}
    Let $\gamma_{q,w}:[0,b]\to M$ be a null geodesic. Then $p=\gamma_{q,w}(b)$ is a null conjugate point if and only if $\exp_q:L_qM\to M$ is singular at $bw$, i.e. if there exists a nonzero $\xi\in T_{bw}(L_qM)$ such that $d\exp_q(\xi)=0$.
\end{proposition}
\begin{proof}
    We begin by proving the backwards direction and to that end assume that there exist a nonzero $\xi\in T_{bw}(L_qM)$ such that $d\exp_q(\xi)=0$. By the construction of the tangent space there thus exists a non-constant path $\xi:(-\varepsilon,\varepsilon)\to L_qM$ with $\xi(0)=bw$. This allows us to construct the variation $\x(u,v)=\exp_q(\frac{u}{b}\xi(v))$ which has $\x(t,0)=\gamma_{q,w}(t)$ and is a variation through null geodesics. Finally we have $\x_v(b,0)=d\exp_q(\xi)=0$ by the chain rule.

    For the other direction we first note that by definition $\x(u,v)=\exp_q(u\x_u(0,v))$ and $\x_u(0,v)\in L_qM$ as $\x$ is a variation through \emph{null} geodesics. 
    Now again by the chain rule we have $0=\x_v(b,0) = d\exp_q\rvert_{bw} \circ \frac{d}{dv}(bx_u(0,v))\rvert_{v=0}$. But since $\xi := \frac{d}{dv}(bx_u(0,v))\rvert_{v=0} \in T_{bw}(L_qM)$ we are done.
\end{proof}

Null conjugate points are also conformal invariants:
\begin{proposition}
    Let $\Phi:(M,g)\to (N,h)$ be a conformal diffeomorphism and $\gamma:[0,b]\to M$ a null geodesic. Then $\gamma(b)$ is a null conjugate point of $\gamma$ if and only if $\Psi(\gamma(b))$ is a null conjugate point of $\Psi \circ \gamma$.
\end{proposition}
\begin{proof}
    ((Cite relevant prop))
    Because of the symmetry of the situation we only need to prove one direction and suppose that $\gamma(b)$ is a null conjugate point of $\gamma$. We thus have a variation $\x$ of $\gamma$ through null geodesics. But since $\Phi$ maps null geodesics to null geodesics, $\Phi\circ \x$ is a variation of $\Phi \circ \gamma$ through null geodesics in $N$, which implies that $\Phi(\gamma(b))$ is a null conjugate point of $\Phi \circ \gamma$.
\end{proof}

\section{Geometry of the Light Cone Observations}
From now on $(M,g), K, V, p^+,p^-$ be as in theorem \ref{thm:intreconstr} (we suppress the indices to simplify notation). ((More explanation))

\subsection{Parameterization of Observations}
\begin{lemma}\label{lem:Kcharact}
    We have:
    \begin{enumerate}[label={\textnormal{(\arabic*)}}]
        \item $(J^-(p^+)\setminus I^-(p^+)) \cap K = \mc{L}^-_{p^+} \cap K$ and thus $K = \mc{L}^-_{p^+} \cap J^+(p^-)$.
        \item There exists a surjective smooth map $\Theta:S^{n-1}\times[0,1] \to K$ such that the curves $\mu_a:=t\mapsto\Theta(a,t), a\in S^n$ are null geodesics, 
        \[\Theta(S^{n-1}\times\{1\}) = \{p^+\}, \quad R:=\Theta(S^{n-1}\times\{0\}) = (J^-(p^+)\setminus I^-(p^+)) \cap J^+(p^-)\]
        and $\Theta:S^{n-1}\times[0,1)\to K\setminus p^+$ is a diffeomorphism.
        \item $\mc{L}^-_{p^+}\cap \interior{\jp} = \emptyset$ and $\mc{L}^-_{p} \cap \interior{\jp} = \emptyset \quad \forall p\in R$.
        % \item There exist $0<t_-<t_+<1$ such that the restriction $\Psi\rvert_{S^n\times[t_-,t_+]}$ is a diffeomorphism onto its image and that for all $v \in S^n$, we have 
        % \[
        % \Psi(v,t_-) \notin \bigcup_{q\in \overline{V}} J^+(q), \quad \Psi(v,t_+) \in \bigcap_{q\in \overline{V}} J^+(q).
        % \]
    \end{enumerate}
\end{lemma}
\begin{proof}
(1) follows immediately from the fact that no past-pointing null geodesic starting at $p^+$ has a cut point in $K$.

For (2) we first note that again, because no past-pointing null geodesic starting at $p^+$ has a cut point in $K$, $exp_{p^+}:exp_{p^+}^{-1}(K)\subset L^-_{p^+}M\to K$ is a diffeomorphism away from $0$. We can smoothly parametrize $exp_{p^+}^{-1}(K)$ to obtain $\Phi:S^{n-1}\times[0,1]\to K$ satisfying (2) ((elaborate))

Finally we prove (3): As $\interior{\jp} = I(p^-,p^+)$, $p\in \mc{L}^-_{p^+}\cap \interior{\jp}$ would imply there exists a timelike path from $p to p^+$. Thus $p$ cannot be in $K$. Now either 

((Overhaul))
As we have no cut point in $J(p_1^-,p_1^+)$, the exponential map at $p_1^+$ is a diffeomorphism onto $J(p_1^-,p_1^+)$. Thus the preimage $\exp^{-1}_{p_1^+}(R)$ of the smooth submanifold
\[
    R = (J^-(p_j^+)\setminus I^-(p_j^+)) \cap (J^+(p_j^-)\setminus I^+(p_j^-)) = \mc{L}^-_{p_1^+} \cap \mc{L}^+_{p_1^-}
\]
is a smooth submanifold of $L^-_{p^+_1}M$. We then let $\mc{A}=R$ and denote by $\mu_a(s) = \gamma_{p_1^+,a}(1-s)$ for $a\in R$. It is then easily checked that this parameterization satisfies all requirements and we are done.
\end{proof}

Note that this implies that $K$ is a smooth $n$-dimensional submanifold of $M$ at any point away from its boundary. We will often treat $K$ itself as a submanifold when it is clear that we are working away from the boundary. This is often the case as by (3) no null geodesic originating from the interior of $\jp$ can reach $p^+$ or $R$, i.e. the boundary of $K$.

% The next proposition allows us to endow $K$ with a number of \enquote{laboratory frames} we will use to conveniently describe the light cone observations on $K$.
% \begin{proposition}[Laboratory Frames]
% Let $(M_j,g_j), K_j, V_j, p_j^+,p_j^-, \Phi$ be as in the statement of theorem \ref{thm:babycase}
% Then there exists a family of future pointing, null geodesics $\mu_a^{(1)}:[0,1]\to K_1$ indexed by $a\in \mathcal{A}$ where $\mathcal{A}$ is a metric space. Furthermore we can require the map $[0,1]\times\mathcal{A}\to K_1; (s,a)\mapsto \mu^{(1)}_a(s)$ to be open ((almost, needed?)) and continuous. If we then take $\mu^{(2)}_a\coloneqq\Phi(\mu^{(1)}_a)$ we can achieve
% \begin{equation}\label{eq:frameunion}
% K_j = \bigcup_{a\in \mathcal{A}}\mu^{(j)}_a([0,1]).
% \end{equation}
% \end{proposition}
% \begin{proof}
% ((TODO))
% \end{proof}

% \begin{remark}\label{rmk:data}To simplify notation we will continue with the construction on just one Lorentzian manifold $(M,g)$ of dimension $1+n$ and assume that we are given the following data to construct the required conformal diffeomorphism ((explain better)) from theorem \ref{thm:babycase}.

% \begin{enumerate}
%     \item A the quasi-manifold $K$,
%     \item the conformal class of $g\rvert_K$ (but not only restricted to tangent vectors in $K$ ((i think??)) ),
%     \item the paths $\mu_a:[0,1]\to K, a\in \mathcal{A}$,
%     \item the set $\mathcal{P}_K(V)$ where $V$ is open and $\overline{V}\subset \operatorname{int} J(p^-,p^+)$ is compact.
% \end{enumerate}
% Note that these data are invariant under conformal diffeomorphism, and any map we construct from it will thus also be invariant. We also remark that $\overline{V}\subset \operatorname{int} J(p^-,p^+)$ implies that $q\notin K$ for any $q\in \overline{V}$. 
% \end{remark}

\subsection{Geometry of Light Observation Sets}


\begin{lemma}\label{prop:transversality}
For any $q\in \overline{V}$ the restriction of the exponential map to null vectors $\exp_q:L^+_qM\to M$ is \emph{transverse} to $K$, i.e. for all $w\in L^+_qM$ such that $\gamma_{q,w}(1) = p\in K$ we have $\gamma'_{q,w}(1)\notin T_pK$.
\end{lemma}
\begin{proof}
    We first establish that $\mc{L}^-_{p^+} \cap \overline{V} = \emptyset$. By remark \ref{rmk:data} we have $\overline{V}\cap K = \emptyset$. By lemma \ref{lem:Kcharact}(1) we know that $K=\mc{L}^-_{p^+}\cap J^+(p^-)$. This means any intersection of $\overline{V}$ and $\mc{L}^-_{p^+}$ must occur in $\mc{L}^-_{p^+} \setminus J^+(p^-)$. But since $\overline{V}$ lies entirely within $J^+(p^-)$ this is also impossible and $\mc{L}^-_{p^+} \cap \overline{V}$ must be empty. 

    In order to achieve a contradiction we now assume that there exists a $q\in \overline{V}$ and a $w\in L^+_qM$ such that with $v:=\gamma_{q,w}(1)\in L_pK$.
    Since $K$ is generated by backwards null geodesics originating at $p^+$ there exists a $u\in L^-_{p^+}M$ such that there exists a $t\in \R_+$ with $\gamma_{p^+,u}(t)=p, \gamma'_{p^+,u}(t)=-v$. We can thus obtain an unbroken past-pointing null geodesic from $p^+$ to $q$ by connecting $\gamma_{p^+,u}$ and $\gamma_{p,-v}$. But this implies that $q\in \mc{L}^-_{p^+}$ which is a contradiction to our previous fact.

    Finally we prove that this implies that $\exp_q:L^+_qM\to M$ is transverse to $K$, i.e. we need to prove that for every $w\in L^+_qM$ with $\exp_q(w)=p\in K$ we have 
    \[
        \im(d\exp_q\rvert_w) \oplus T_pK = T_pM.
    \]
    As $T_pK$ is a null hypersurface we only need to prove that $\im(d\exp_q\rvert_w)$ contains a null vector which is not a multiple of the null vector $v\in T_pK$ generating $T_pK = v^\perp$. But by the properties of the exponential map, $\im(d\exp_q\rvert_w)$ contains $v' = \gamma'_{q,w}(1) \in T_pM$. And since we just proved that $v'\notin T_pK$, $v+v'$ must be a timelike vector and $\im(d\exp_q\rvert_w) \oplus T_pK = T_pM$, as desired.
\end{proof}

\begin{lemma}\label{lem:hitsonce}
    For $q\in \overline{V}$ and $w\in L^+_qM$ there exists exactly one $t\in (0,\infty)$ such that $\gamma_{q,w}(t)\in K$.
\end{lemma}
\begin{proof}
    Let $q\in \overline{V}$ and $w\in L^+_qM$, by ((Leavescompact)) any geodesic starting in the compact set $J(p^-,p^+)$ must eventually leave it, intersecting the boundary. Thus there exists at least one $t\in (0,\infty)$ with $p=\gamma_{q,w}(t)\in K$. ((Do compactness argument?)) We WLOG assume that $t$ is the smallest such value. Now, by the previous lemma, we have $\gamma'_{q,w}(t) \notin T_pK$. ((But now we are outside $J^-(p^+)$))
\end{proof}

\begin{lemma}[Direction Reconstruction]\label{lem:dirreconstr}
    ((Iso = Bijection))
Let $p\in K$ then there exists an isomorphism $\Phi$ between the space $\mc{S}$ of spacelike hyperplanes $S\subset T_pK$ and the space $\mc{V}$ of rays $\R_+V\subset T_pM$ along future-directed outward facing null vectors, given by the mapping $S\in \mc{S}$ to the unique future-directed outward pointing null ray $\Phi(S)$ contained in $S^\perp$. The inverse map is given by $\mc{V}\ni \R_+V \mapsto T_pK \cap V^\perp\in \mc{S}$.

Moreover there exists an isomorphism between $\mc{S}$ and the space $\mc{N}$ of linear null hypersurfaces $N\subset T_pM$ which contain a future-directed outward pointing null vector given by $\mc{S}\mapsto S\oplus \operatorname{span} \Phi(S)\in \mc{N}$.
\end{lemma}
\begin{proof}
    Let $p\in K$, and $S\subset T_pK$ be a spacelike hyperplane. The orthogonal complement $S^\perp\subset T_pM$ then is a two-dimensional lorentzian subspace. There thus exist four light rays $V,-V,W,-W$ in $S^\perp$. Since $T_pK=v^\perp$ for some future-pointing null vector $v\in T_pK$, we have $v\in S^\perp$ and can WLOG assume $V=v$. This leaves $W$ as the unique future-pointing outward null ray which is perpendicular to $S$, and we can thus set $\Phi(S)=W$.

    For the second part, we let $0\neq V\in T_pM$ be an outward future-pointing null vector. In particular this means that $V\notin T_pK$. Thus $S=V^\perp\cap T_pK$ is a spacelike hyperplane in $T_pK$ which satisfies $S=\Phi^{-1}(V)$. ((... do isomorphism fun?))

    For the final claim we note that ((... Why iso?))
\end{proof}

\begin{definition}[Observation Preimage]
For any $q\in \interior{\jp}$ with light observation set $\mc{P}_K(q)\subset K$ we define the \emph{observation preimage} $L^K_qM$ to be the preimage of $K$ under the exponential map restricted to $L^+_qM$, i.e. 
\begin{align*}
    L^K_qM := (\exp_q\rvert_{L^+_qM})^{-1}(K) \subset L^+_qM
\end{align*}
\end{definition}
\begin{lemma}\label{lem:preimage}
For any $q\in \interior{\jp}$, the observation preimage $L^K_qM$ is a $n-1$-dimensional submanifold of $T_pM$. 

Furthermore, for any $w\in L^K_qM$ there exist a relatively open neighborhood $\mc{W}\subset L^K_qM$ such that $\exp_q:\mc{W}\to \exp_q(\mc{W})\subset \mc{P}_K(q)$ is a diffeomorphism.
\end{lemma}
\begin{proof}
    By lemma \ref{prop:transversality}, $\exp_q:L^+_qM\to M$ is transverse to $K$ (here we treat $L^+_qM$ and $K$ as submanifolds as the points where they fail to be submanifolds can be removed without impacting the proof). Thus by the preimage lemma $ L^K_qM := (\exp_q\rvert_{L^+_qM})^{-1}(K)$ is a $n-1$-dimensional submanifold of $L^+_qM$.

    For the second part let $w\in L^K_qM$, since $p:=\exp_q(w)\in K$ and we assumed that such a $p$ cannot be a null conjugate point, we know that $\exp_q:L_qM\to M$ has an invertible differential at $w$. Thus, by the implicit function theorem, there exists an open neighborhood $\mc{W}'\subset L_qM$ of $w$ such that $\exp_q:\mc{W}'\to \exp_q(\mc{W}')$ is a diffeomorphism. If we then restrict $\exp_q$ to $\mc{W}:=\mc{W}' \cap L^K_qM$ the map is still a diffeomorphism as desired.
\end{proof}

\begin{lemma}\label{lem:finitevecs}
    Let $q\in \interior{\jp}$ and $p\in \mc{P}_K(q)$ then there exist only finitely many $v_1,\dots,v_n\in L^K_qM$ such that $\exp_q(v_i)=p$. Furthermore for any neighborhood $W\subset L^K_qM$ of $v_1,\dots,v_n$, there exists a neighborhood $U\subset \mc{P}_K(q)$ of $p$ such that $\exp^{-1}_q(U) \cap L^+_qM \subset W$. 
\end{lemma}
\begin{proof}
    ((Overhaul))

    Let $q\in \overline{V}$ and $v\in L^+_qM$ such that $p=\exp_q(v)$. Since we required that $p\in K$ cannot be a null conjugate point of $q$, $\exp_q$ must be a local diffeomorphism around $v$. This means that there exist open sets $v\in\mc{O}_v\subset L_qM, p\in \mc{U}_v\subset M$ such that $\exp_q:\mc{O}_v\to \mc{U}_v$ is a diffeomorphism. But this means that there cannot exist another $v'\in \mc{O}_v$ with $\exp_q(v')=p$. We now restrict ourselves only to null \emph{directions} at $q$ i.e. the quotient $L^+_qM/\R_+ \simeq S^{n-1}$. Since any null vector $v$ with $\exp_q(v)=p$ has an open neighborhood where no other vector can have this property, the set of null \emph{directions} in $S^{n-1}$ which hit $p$ is discrete and thus finite because $S^{n-1}$ is compact. Because we only have finitely many null directions which hit $p$, $\pi^{-1}(p)\cap \DP(q)$ can only have finitely many elements, as desired.
\end{proof}


We can immediately put these lemmas to use and prove this proposition characterizing the light observation set.
\begin{proposition}\label{prop:unionmanif}
Let $q\in \interior{\jp}$ and $p\in \mc{P}_K(q)$. There exists a neighborhood $\mc{O}$ of $p$, a positive integer $N$ and $N$ pairwise transversal, spacelike,  codimension 1 submanifolds $\mc{V}_i\subset K$ such that $\mc{P}_K(q) \cap \mc{O} = \bigcup_{i=1}^N \mc{V}_i$ and $p\in \mc{V}_i$ for $i={1,\dots, N}$.
\end{proposition}
\begin{proof}
    Let $q\in \overline{V}$ and $p\in \mc{P}_K(q)$. By the previous lemma we know that there can only be finitely many $w_1,\dots,w_n\in L^K_qM$ with $\exp_q(w_i)=p$. 

    By lemma \ref{lem:preimage}, for each $w_i$ there exists a neighborhood $\mc{W}_i\subset L^K_qM$ of $w_i$ such that $\exp_q:\mc{W}_i\to \mc{V}_i:=\exp_q(\mc{W}_i)$ is a diffeomorphism. Thus $\mc{V}_i\subset \mc{P}_K(q)$ is a submanifold of $K$ and we have $\bigcup_{i=1}^N\mc{V}_i\subset \mc{P}_K(q)$. 

    Now we use the second part of the previous lemma to obtain an open neighborhood $\mc{O}\subset \mc{P}_K(q)$ of $p$, such that $\exp_q^{-1}(\mc{O})\cap L^+_qM\subset \bigcup_{i=1}^N \mc{W}_i$. Thus any point $p\in \mc{P}_K(q) \cap \mc{O}$ is contained in some $\mc{V}_i$ and we have $\bigcup_{i=1}^N\mc{V}_i \supset \mc{P}_K(q)\cap \mc{O}$. After possibly shrinking some $\mc{W}_i$ we get equality.

    The fact that each $\mc{V}_i$ is spacelike follows as it can be written as the intersection of two transversal null hypersurfaces, $\mc{L}^+_q$ and $K$.

    Finally to prove that they are transversal at $p$, we assume by contradiction that there exist $i\neq j$ such that $T_p\mc{V}_i=T_p\mc{V}_j$. But by lemma \ref{lem:dirreconstr} this would imply that $v_i = c*v_j$ for a $c\in \R_+$, where $v_i = \gamma'(1)_{q,w_i}$. Thus we would have $w_i = w_j$, a contradiction.
\end{proof}

\begin{definition}[Regular Point]
We call a point $p\in \mc{P}_K(q)$ \emph{regular} if there exists an open neighborhood $\mc{O}\subset M$ of $p$ such that $\mc{O}\cap \mc{P}_K(q)$ is a submanifold.
\end{definition}


\begin{corollary}The subset of regular points is open and dense in $\mc{P}_K(q)$.
\end{corollary}
\begin{proof}
It suffices to show that for every cut point $p\in \mc{P}_K(q)$, every relatively open neighborhood $\mc{O}\subset\mc{P}_K(q)$ contains a regular point. By the previous proposition, for $\mc{O}$ small enough we have $\mc{P}_K(q)\cap \mc{O} = \bigcup_{i=1}^N\mc{V}_i$, where $\mc{V}_i$ are pairwise transversal. This means their intersection is of lower dimension and we can find a $p'\in \mc{V}_i$ for some $i\in 1,\dots,N$ such that $p'\notin \mc{V}_j$ for $j\neq i$. Thus we can find an open neighborhood $\mc{O}'$ around $p'$ such that $\mc{O}'\cap \mc{P}_K(q)\subset \mc{V}_i$ which means $p'$ is a regular point, as desired.
\end{proof}

\subsection{Observation Time Functions}
\begin{definition}[Observation Time Function]\label{def:observationtime}
For $a\in \sn$ the \emph{observation time function} 
is defined as 
\begin{align*}
    f_a:\jp&\to [0,1]\\
    q&\mapsto\inf(\{s\in [0,1] \mid \mu_a(s)\in J^+(q)\}\cup \{1\}).
\end{align*}
Moreover, let $\mc{E}_a(q)\coloneqq\mu_a(f_a(q))\in M$ be the earliest point where $\mu_a$ sees light from $q$.
\end{definition}

\begin{lemma}\label{lem:observationtime}
Let $a\in \sn$ and $q \in \interior{\jp}$. Then
\begin{enumerate}[label={\textnormal{(\arabic*)}}]
    \item It holds that $f_a(q)\in (0,1)$.
    \item We have $\mc{E}_a(q)\in J^+(q)$ and $\tau(q,\mc{E}_a(q))=0$. Moreover the function $s\mapsto\tau(q,\mu_a(s))$ is continuous, non-decreasing on $[0,1]$ and strictly increasing on $[f_a(q),1]$.
    \item Let $p\in K$. Then $p=\mc{E}_a(q)$ with some $a\in \ca$ if and only if $p\in \mathcal{P}_K(q)$ and $\tau(p,q)=0$. Furthermore, these are equivalent to the fact that there are $v\in L^+_qM$ and $t\in[0,\rho(q,v)]$ such that $p=\gamma_{q,v}(t)$.
\end{enumerate}
\end{lemma}
\begin{proof}
Let $a\in \mc{A}$ and $q\in \overline{V}$.

We begin by showing (1): By lemma \ref{lem:Kcharact}(3) we have that
$\mu_a(t_-)\notin J^+(q)$ and $\mu_a(t_+)\in J^+(q)$. The second part immediately yields $f_a(q) \leq t_+$ as $f_a(q)$ is the infimum over all observation times. For the first part we assume by contradiction that there were to exist a $t_{-2}<t_-$ with $\mu_a(t_{-2})\in J^+(q)$. This allows us to construct a causal path from $q$ to $\mu_a(t_-)$ by joining the causal path from $q\to \mu_a(t_{-2})$ and the null geodesic $\mu_a$ from $t_{-2}$ to $t_-$. Since this would imply that $\mu_a(t_-)\in J^+(q)$ this is a contradiction and $f_a(q)$ must be bigger than $t_-$ proving (1).

(2)
By the definition of the infimum we can find a sequence $t_n\searrow f_a(q)$ such that for all $t_n$ we have $\mu_a(t_n)\in J^+(q)$. Now since $t\mapsto \mu_a(t)$ is continuous we have that $\mu_a(t_n)\to \mu_a(f_a(q)) = \mc{E}_a(q)$. Since $J^+(q)$ is closed this yields $\mc{E}_a(q)\in J^+(q)$. 

For the second part we assume by contradiction that $\tau(q,\mc{E}_a(q)) > 0$. Since this means that a timelike path from $q$ to $\mc{E}_a(q)$ exists we have $\mc{E}_a(q)\in I^+(q)$. Then, since $I^+(q)$ is open we can find a $t<f_a(q)$ such that $\mu_a(t)\in I^+(q) \subset J^+(q)$. This is a contradiction since $f_a(q)$ is the infimum over such $t$.

To show that $s\mapsto \tau(q,\mu_a(s))$ is continuous and non-decreasing on $[0,1]$ we first note that it is the composition of two continuous functions. Non-decreasing then follows from the reverse triangle inequality together with the fact that $\mu_a$ is a null path.

Finally to show that $s\mapsto \tau(q,\mu_a(s))$ is strictly increasing in $[f_a(q),1]$ we let $f_a\leq t_1<t_2\leq 1$. Now by ((REF)) there exists a causal geodesic $\gamma_1:[0,1]\to M$ with $\gamma_1(0)=q$ and $\gamma_1(1)=\mu_a(t_1)$ such that $L(\gamma_1)=\tau(p,\mu_a(t_1))$. 
If we then connect $\gamma_1$ to $\mu_a\rvert_{[t_1,t_2]}$ we get a path $\gamma_2$ connecting $q$ to $\mu_a(t_2)$ which has length $L(\gamma_2) = L(\gamma_1)$ as $\mu_a$ is a null geodesic. Next we argue that $\gamma_2$ must have a break at the connecting point, i.e. $\gamma_1'(1) \neq c\mu_a'(t_1)$ for any $c\in \R_+$. If $\gamma_1$ is timelike this observation is trivial as $\mu_a$ is lightlike. If however, $\gamma_1$ is lightlike (which is exactly the case if $t_1=f_a(1)$), this fact follows from the transversality of light cone observations as noted in proposition \ref{prop:transversality}. This means that $\gamma_2$ is a broken causal geodesic, which by ((REF)) implies that there exists a strictly longer timelike path $\gamma_3$ connecting the endpoints and we get
\[
\tau(q,\mu_a(t_2)) \geq L(\gamma_3) > L(\gamma_2) = L(\gamma_1) = \tau(q,\mu_a(t_1)).
\]

Next to prove (3):
To prove the fist direction we assume that $p=\mc{E}_a(q)$ for some $a\in \mc{A}$. Now by (2) we have $\mc{E}_a(q) \in J^+(q)$ and $\tau(q,\mc{E}_a(q))=\tau(q,p)=0$. But now, by ((REF)) there exists a null geodesic from $q$ to $p$ which means $p\in \mc{P}_K(q)$. 

For the other direction we let $p\in \mc{P}_K(q)$ with $\tau(q,p)=0$. Now let $a\in \mc{A}$ such that $p=\mu_a(t)$ for some $t\in [0,1]$. We then assume by contradiction that $\mc{E}_a(q) \neq p$, i.e. $f_a(q) < t$. But by (2) we have that $s\mapsto\tau(q,\mu_a(s))$ is strictly increasing after $f_a(q)$ which is in contradiction with $\tau(q,p)=0$.

The other equivalence follows the definition of $\mc{P}_K(q)$ together with the definition of cut points.

Finally we prove (4):
Let $q_i\to q$ in $\overline{V}$, let $t_i = f_a(q_i)$ and $t=f_a(q)$. Since $\tau$ is continuous, for any $\varepsilon>0$ we have $\lim_{j\to \infty} \tau(q_j,\mu_a(t+\varepsilon)) = \tau(q,\mu_a(t+\varepsilon)) > 0$. Thus for $i$ big enough we have $\tau(q_i,\mu_a(t+\varepsilon))>0$. But by (3) this implies that $a$ must have observed $q_i$ before $t+\varepsilon$ i.e. $f_a(q_i)<t+\varepsilon = f_a(q) + \varepsilon$. As $\varepsilon$ was arbitrary we get $\limsup_{j\to \infty} t_j \leq t$.

We assume now that $\liminf_{i\to \infty} t_i=t'<t$. Let $(t_i)$ be a convergent subsequence such that $f_a(q_i) = t_i \to t' < f_a(q)$. Now by the continuity of $\tau$ and $\mu_a$ we have 
\[
0=\tau(q_i,\mu_a(f_a(q_i)))\to \tau(q,\mu_a(t')).
\]
Furthermore by ((REF)) $\mu(s_i)\in J^+(q_i)$ for all $i$ implies $\mu(s')\in J^+(q)$. But now we have $\mu(s')\in \mc{P}_K(q)$ and $\tau(q,\mu_a(s'))=0$ which by (3) implies that $\mu_a(s') = \mc{E}_a(q) = \mu_a(f_a(q))$. But this is a contradiction as $s'<f_a(q)$. ((More in-detail?))
\end{proof}

By (3) of the above lemma, for any $q\in \overline{V}$ and $a\in \ca$ we have $\mc{E}_a(q)\in \P_K(q)$. Since $\P_K(q)\subset J^+(q)$, we can see using definition \ref{def:observationtime} that the set of earliest observations $\P_K(q)$ and the path $\mu_a$ completely determine the functions
\begin{align}\label{eq:observationtimerestated}
    f_a(q) = \min \{ s\in [-1,1] \mid \mu_a(s)\in \P_U(q) \}, \quad \mc{E}_a(q) = \mu_a(f_a(q))
\end{align}

\begin{lemma}\label{lem:qcont} Let $a\in \sn$. Then the function $q\mapsto f_a(q)$ is continuous on $\interior{\jp}$.
\end{lemma}

\begin{lemma}\label{lem:acont} Let $q\in \interior{\jp}$. Then the function $a\mapsto f_a(q)$ is continuous on $\sn$.
\end{lemma}

\begin{corollary}\label{cor:fcont} 
    The map $f:\interior{\jp}\times\sn\to\R; (q,a)\mapsto f_a(q)$ is continuous.
\end{corollary}

\begin{corollary}\label{cor:funif}
    If $q_n \to q_0\in \interior{\jp}$ as $n\to \infty$ and we denote $F_q:\sn\to\R;a\mapsto f_a(q)$. Then $F_{q_n}\to F_{q_0}$ uniformly over $\sn$ as $n\to \infty$.
\end{corollary}


\subsubsection{Set of earliest observations}
\begin{definition}[Set of earliest observations]
For $q\in \overline{V}$ we define
\begin{alignat*}{2}
    \mathcal{D}_K(q) &=\; &\{(p,v)\in L^+M \mid &(p,v) = (\gamma_{q,w}(t),\gamma'_{q,w}(t)) \\
    &&&\text{ where } p\in K, w\in L_q^+M, 0\leq t \leq \rho(q,w)\},\\
    \mathcal{D}^{reg}_K(q) &=\; &\{(p,v)\in L^+M \mid &(p,v) = (\gamma_{q,w}(t),\gamma'_{q,w}(t)) \\
    &&&\text{ where } p\in K, w\in L_q^+M, 0 < t < \rho(q,w)\},
\end{alignat*}
We say that $\mathcal{D}_K(q)$ is the \emph{direction set} of $q$ and $\mathcal{D}^{reg}_K(q)$ is the \emph{regular direction set} of $q$.

Let $\mc{E}_K(q) = \pi(\mathcal{D}_K(q))$ and $\mc{E}^{reg}_K(q) = \pi(\mathcal{D}^{reg}_K(q))$, where $\pi:TM\to M$ is the canonical projection. We say that $\mc{E}_K(q)$ is the set of earliest observations and $\mc{E}^{reg}_K(q)$ is the set of earliest regular observations of $q$ in $K$. We denote the collection of earliest observation sets by $\mc{E}_K(V) = \{ \mc{E}_K(q) \mid q\in V\}$.
\end{definition}

Note that $\mc{E}_K(q) = \{ \mc{E}_a(q) \mid a\in \sn \}$.
\begin{proposition}\label{prop:submanifolds}
For any $q\in \overline{V}$ it holds that
\begin{enumerate}[label={\textnormal{(\arabic*)}}]
    \item $\mc{E}^{reg}_K(q)$ is a $n-1$-dimensional nonempty spacelike submanifold of $K$ which is open relative to $\mc{P}_K(q)$ and has $\overline{\mc{E}^{reg}_K(q)} = \mc{E}_K(q)$ and,
    \item $\mc{E}_K(q)$ fails to be a submanifold exactly at cut points,
    \item $\mc{D}^{reg}_K$ is a nonempty submanifold of $\overrightarrow{K}:=\pi^{-1}(K)$ ((...)) which is open 
\end{enumerate}
\end{proposition}
\begin{proof}
We begin by proving (1):

Let $p$

\end{proof}

Note that since $\mc{E}^{reg}_K(q)$ is exactly $\mc{E}_K(q)$ without the cut points, it is also the collection of all points where $\mc{E}_K(q)$ is locally a submanifold.

\begin{proposition} For any $q\in \overline{V}$, $\mc{E}{reg}_K(q)\subset K$ and $\mathcal{D}^{reg}_K(q)\subset TU$ are smooth submanifolds of dimension $n-1$ ((D has dim $n$)).
\end{proposition}
\begin{proof}
((TODO))


We will focus our attention to the case of $\mc{E}{reg}_U(q)$ as the argument for  $\mathcal{D}^{reg}_U(q)$ is analogous
Note first that $\mc{E}{reg}_U(q)$ can be rewritten as 
\[
    \{\exp_q(w) \mid  w\in L^+_qM \text{ with } 1<\rho(q,w)\}.
\]
Next by lower semi-continuity of $\rho$ we get that $R=\{w\in L^+_qM \mid 1<\rho(q,w)$ is an open set and thus a dimension $(n-1)$ submanifold (this is because $L^+_qM$ itself is of dimension $(n-1)$). But since $\rho(q,w)$ describes where $\exp_q$ first fails to be a diffeomorphism we get that the surjection $\exp_p:R\to \mc{E}{reg}_U(q)$ is a diffeomorphism. Thus, since $R$ was a manifold of dimension $(n-1)$, $\mc{E}{reg}_U(q)$ is also a manifold and has the required dimension.
\end{proof}

Finally in this section we will prove
\begin{proposition}\label{prop:seocharact}
Let $q\in \overline{V}$, then 
\begin{equation*}
    \mc{E}_K(q) = \{ p \in \P_K(q) \mid \text{there are no $p'\in \P_K(q)$ such that $p' < p$} \}.
\end{equation*}
\end{proposition}
\begin{proof}
((Still True?))
For the left inclusion assume $p\in \mc{E}_U(q)$, i.e. there exists an $a\in \ca$ such that $\mc{E}_a(q)=p$. Then lemma \ref{lem:observationtime}(3) immediately yields, $p\in \P_U(q)$ and  $\tau(q,p)=0$. Now suppose there were a $p'\in \P_U(q)$ with $p'\ll p$. By as $\P_U(q)\subset J^+(q)$ we have $q\leq p'$, then as $p'\ll p$ we get $q\ll p$. But this would imply $\tau(p,q)>0$, a contradiction.

For the other direction we assume we have $p \in \P_U(q)$ such that there are no $p'\in\P_U(q)$ such that $p'\ll p$. Again by lemma \ref{lem:observationtime}(3) we only need to prove that $\tau(p,q)=0$. Suppose that $\tau(p,q)>0$. By equation \ref{eq:frameunion} there exists an $a\in \ca$ and a $s\in [-1,1]$ such that $\mu_a(s) = p$. Now since $\tau(p,q)>0$, we must have $s > f_a(q)$. But then $\mc{E}_a(q) = \mu_a(f_a(q)) \ll  \mu_a(s)$, since $\mu_a$ is timelike, which is a contradiction.
\end{proof}
Thus $\mc{E}_U(q)$ truly deserves to be called the \enquote{set of earliest observations}.

